\chapter{Droite affine de matrices non inversibles}

\section{Sujet}

\paragraph{Exercice}
Soit $n \geqslant 2$ un entier et $A,B \in \mathscr M_n(\mathbb R)$. Soit $(t_i)_{1\leqslant i \leqslant n+1}$ des nombres réels différents. Montrer que les deux assertions suivantes sont équivalentes :
\begin{enumerate}
    \item[(a)] pour tout $1 \leqslant i \leqslant n+1,\ \det(A+t_iB) = 0$
    \item[(b)] il existe $V,W$ deux sous espaces vectoriels de $\mathbb R^n$ tels que $A(V) \subset W,\ B(V) \subset W$ et $\dim W < \dim V$.
\end{enumerate}

\paragraph{Deuxième exercice}
On considère
$$
f(x) = \sum_{k=0}^{\infty} f_k x^k,
\qquad
g(x) = \sum_{k=0}^{\infty} g_k x^k
$$
deux séries entières à coefficients strictement positifs de rayons de convergence respectifs $r_f$ et $r_g$.
On suppose que $0 < r_f < r_g$ et que la suite $f_n/f_{n+1}$ converge.
Montrer qu'il existe $a,b > 0$ tels que pour tout $n \geqslant 1$ on a $g_n \leqslant a f_n e^{-bn}$.

\section{Solution de l'exercice}


Supposons qu'on dispose de deux sous-espaces vectoriels $V,W$ tels que prescrits par l'assertion (b). Pour tout $t \in \mathbb R$, on obtient alors $(A+tB)(V) \subset W$ et le théorème du rang montre donc $V \cap \operatorname{Ker}(A+tB)$ est de dimension minorée par $1$. En particulier, $A+tB$ n'est pas inversible et son déterminant est donc nul.

Supposons que (a) soit vérifiée. La fonction $t \mapsto \det(A + tB)$ étant polynomiale, de degré majoré par $n$, elle est alors nulle. Autrement dit, les matrices $A+tB$ sont non inversibles. Un exemple simple est le cas où $\operatorname{Ker}(A) \cap \operatorname{Ker}(B)$ est de dimension minorée par $1$. L'assertion (b) est alors satisfaite en prenant pour $V$ cette intersection et $W = \{0\}$. Supposons qu'on ne soit pas dans cette situation :  la matrice $B$ induit alors par restriction un isomorphisme de $\operatorname{Ker}(A)$ sur $B(\operatorname{Ker(A)})$.
Soient $E'$ et $F'$ des supplémentaires respectifs de ces deux sous-espaces vectoriels dans $\mathbb R^n$. En considérant des bases adaptées aux décompositions $\mathbb R^n = \operatorname{Ker}(A)\oplus E'$ et $\mathbb R^n = B(\operatorname{Ker}(A))\oplus F'$, on dispose de matrices de passage $P,Q$  telles que :
\[
PAQ = \begin{pmatrix}0 & *\\0 & A'\end{pmatrix},
\qquad
PBQ = \begin{pmatrix}I & *\\0 & B'\end{pmatrix},
\]
où $A',B'$ sont des matrices de $\mathscr M_{n'}(\mathbb R)$ avec $n' = \dim(E') = \dim(F') < n$ car $\dim(\operatorname{Ker}(A)) \geqslant 1$.
On constate alors que $\det(A' + t B') = 0$ pour tout $t\in\mathbb R$, ce qui permet de conclure par récurrence (le cas $n = 1$ est trivial). En effet, si $V',W'$ sont des sous-espaces vectoriels respectifs de $E'$ et $F'$ tels que $A'(V') \subset W'$, $B'(V') \subset W'$ et $\dim(W') < \dim(V')$, alors $A(V') \subset B(\operatorname{Ker}(A)) \oplus W'$ et de même pour $B(V')$, donc il suffit de poser $V = \operatorname{Ker}(A) \oplus V'$ et $W = B(\operatorname{Ker}(A)) \oplus W'$.

\paragraph{Remarque} \textit{Via} le choix des bases, on identifie ici les matrices $A'$ et $B'$ avec les applications linéaires $x \mapsto \pi(A(x))$ et $x \mapsto \pi(B(x))$ de $E'$ vers $F'$, où $\pi$ est la projection sur $F'$ parallèlement à $B(\operatorname{Ker}(A))$. Le concept sous-jacent serait celui de passage au quotient, malheureusement hors-programme en CPGE.

\section{Solution du deuxième exercice}

Sous les conditions de l'énoncé, on montre classiquement (règle de d'Alembert) que la limite de $f_n/f_{n+1}$ est $r_f$.
Soient $r,R$ deux réels tels que $r_f < r < R < r_g$. Puisque
$$
\frac{f_n r^n}{f_{n+1}r^{n+1}} \xrightarrow[n\to\infty]{} \frac{r_f}{r} \in \left]0;1\right[,
$$
la suite $(f_n r^n)_{n\in\mathbb N}$ est croissante à partir d'un certain rang. Elle admet en particulier un minimum (qui est strictement positif).
Par ailleurs, la suite $(g_n R^n)_{n\in\mathbb N}$ est majorée par définition du rayon de convergence. Compte tenu de ce qui précède, on en déduit l'existence d'un réel $a > 0$ tel que :
$$
a = \sup_{n\in\mathbb N} \frac{g_n R^n}{f_n r^n}.
$$
Enfin, on dispose bien d'un réel $b > 0$ tel que $r/R = e^{-b}$ puisque $0 < r < R$.
Le résultat demandé est alors établi.

\paragraph{Remarque}
Puisqu'on peut prendre $r$ et $R$ arbitrairement proches de $r_f$ et $r_g$ respectivement, on constate que l'ensemble des $b > 0$ tels que $\sup_n (e^{bn} g_n/f_n) < \infty$ admet pour borne supérieure $\ln(r_g/r_f)$.
