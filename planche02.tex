\chapter{}

\section{Sujet}

\paragraph{Exercice}
Soit $n \geqslant 2$ un entier et $A,B \in \mathscr M_n(\mathbb R)$. Soit $(t_i)_{1\leqslant i \leqslant n+1}$ des nombres réels différents. Montrer que les deux assertions suivantes sont équivalentes :
\begin{enumerate}
    \item[(a)] pour tout $1 \leqslant i \leqslant n+1,\ \det(A+t_iB) = 0$
    \item[(b)] il existe $V,W$ deux sous espaces vectoriels de $\mathbb R^n$ tels que $A(V) \subset W,\ B(V) \subset W$ et $\dim W < \dim V$.
\end{enumerate}
\paragraph{Deuxième exercice}
On considère
$$
f(x) = \sum_{k=0}^{\infty} f_k x^k,
\qquad
g(x) = \sum_{k=0}^{\infty} g_k x^k
$$
deux séries entières à coefficients strictement positifs de rayons de convergence respectifs $r_f$ et $r_g$.
On suppose que $0 < r_f < r_g$ et que la suite $f_n/f_{n+1}$ converge.
Montrer qu'il existe $a,b > 0$ tels que pour tout $n \geqslant 1$ on a $g_n \leqslant a f_n e^{-b_n}$.

\section{Solution du deuxième exercice}

Sous les conditions de l'énoncé, on montre classiquement (règle de d'Alembert) que la limite de $f_n/f_{n+1}$ est $r_f$.
Soient $r,R$ deux réels tels que $r_f < r < R < r_g$. Puisque
$$
\frac{f_n r^n}{f_{n+1}r^{n+1}} \xrightarrow[n\to\infty]{} \frac{r_f}{r} \in \left]0;1\right[,
$$
la suite $(f_n r^n)_{n\in\mathbb N}$ est croissante à partir d'un certain rang. Elle admet en particulier un minimum (qui est strictement positif).
Par ailleurs, la suite $(g_n R^n)_{n\in\mathbb N}$ est majorée par définition du rayon de convergence. Compte tenu de ce qui précède, on en déduit l'existence d'un réel $a > 0$ tel que :
$$
a = \sup_{n\in\mathbb N} \frac{g_n R^n}{f_n r^n}.
$$
Enfin, on dispose bien d'un réel $b > 0$ tel que $r/R = e^{-b}$ puisque $0 < r < R$.
Le résultat demandé est alors établi.

\paragraph{Remarque}
Puisqu'on peut prendre $r$ et $R$ arbitrairement proches de $r_f$ et $r_g$ respectivement, on constate que l'ensemble des $b > 0$ tels que $\sup_n (e^{bn} g_n/f_n) < \infty$ admet pour borne supérieure $\ln(r_g/r_f)$.
