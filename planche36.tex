\chapter{Processus aléatoire d'ordre 2}

\section{Sujet}

\paragraph{Premier exercice }
On considère la suite $(Y_n)_{n\geqslant 0}$ définie par $\;\: Y_0 =0,\quad Y_1=1,$ et pour $ n\geqslant 2, \:\: Y_n = \vert Y_{n-1} \pm Y_{n-2}\vert$ où les $\pm$ sont choisis aléatoirement de manière indépendante et uniforme. Montrer que:
$$ \mathbb P\big( \forall n \geqslant 1, \: Y_n \neq 0 \big) \in \left]0;1\right[.$$

\paragraph {Deuxième exercice }

\section{Solution du premier exercice}  %LOU16
Notons:$\quad\forall n \in \mathbb N  ,\quad X_n := (Y_{n+1},Y_{n+2})$, et considérons les applications $f,g: \mathbb N^2 \to \mathbb N^2$ définies par 
 $$ f:(a,b)\mapsto (b,a+b), \qquad g:(a,b) \mapsto (b,|a-b|).$$
 $$\forall n \in \mathbb N,\quad \Pr\big[X_{n+1} = f(X_n)\big] =\Pr \big[X_{n+1} = g(X_n)\big] = \dfrac12.$$
  
  Soit $V$ l'ensemble des valeurs, autres que $(0,1)$, que peuvent prendre les $X_n,$ et notons:$\quad\forall u = (a,b)\in V,\:\: |u| = a+b.$
  

 
  
  Alors:$\qquad\forall u \in V \setminus\{(1,0)\}\quad1\leqslant |g(u)|< |u|\quad$ ou $\quad 1\leqslant  |(g\circ g)(u)| < |u|.$
  
  On peut alors définir $ k= \varphi(a,b)$  le plus petit entier tel que $g^k (a,b) =(1,0).$  Soit $Z_n = \varphi (X_n).\quad$ En remarquant que $\:\:\:g\circ g\circ f =\text{Id},$ on a:
  
  $Z_0 =1,\quad Z_{n+1} = Z_n-1$ ou $Z_n +2$ selon que $X_{n+1}= g(X_n)$ ou $ f(X_n).$
  
  Introduisons la suite $(U_n)_{n\in \mathbb N^*}$ de variables aléatoires indépendantes, et égales à la variable $U$ qui prend, de manière équiprobable deux valeurs, $-1$ et $2$, puis, 
  $\:\forall n \in \mathbb N, \:\: S_n = \displaystyle \sum _{i=0}^n U_i.\quad$
  Notons enfin: $\forall k \in \mathbb N,\quad p_k= \displaystyle \Pr \Big[\bigcup_{n=0}^{+\infty} (k + S_n =0)\Big].\quad$ Dans ce contexte: $Z_n= 1+S_n$
  
  $ \displaystyle \Pr\Big [\bigcup _{n=1}^{+\infty} (Y_n =0) \Big]=\Pr\Big[\bigcup _{n=0}^{+\infty} (X_n=(1,0))\Big] =\Pr\Big[\bigcup _{n=0}^{+\infty} (Z_n =0)\Big]= p_1.$
  
  On a: $\quad p_0 =1,\quad \forall k\in \mathbb N^*, $
  
  $p_k = \displaystyle \Pr \Big[ (U_1 =-1) \cap \left(\bigcup_{n=1}^{+\infty} (k +S_n =0)\right)\Big] + \Pr \Big[ (U_1 =2) \cap \left( \bigcup_{n=1}^{+\infty} (k +S_n=0) \right) \Big]$
  
  $p_k= \dfrac 12 \left (\displaystyle \Pr \Big[\bigcup_{n=1}^{+\infty} (k-1+S_{n-1}=0)\Big] + \Pr \Big[ \bigcup _{n=1}^{+\infty} (k+2+S_{n-1} = 0)\Big] \right)$
  
  $$p_k = \dfrac 12 ( p_{k-1} + p_{k+2}).$$
 
  Ainsi, la suite $(p_k)$ vérifie une récurrence linéaire dont le polynôme caractéristique est $(X-1)(X-\alpha)(X- \beta)$ avec $\alpha =\dfrac {-1+\sqrt5}2,\:\: \beta = \dfrac{-1-\sqrt 5}2.$
  
  Il existe des réels $r,s,t$ tels que $\forall k \in \mathbb N, \quad p_k =r+s\alpha ^k+t \beta^k.$
   
   $|\beta| >1,\:\: \displaystyle \lim _{k \to +\infty} \alpha^k =0 $, et le caractère borné de la suite $(p_k)$ font que: 
   
   $t=0,\quad \displaystyle r=\lim_{k\to + \infty} p_k.$
   Alors: $\mathbb E \big( (\frac 34)^{S_n} \big ) = \Big(\mathbb E \left((\frac 34)^{U}\right) \Big)^n = a ^n \:\:$ où 
   
   $a =\dfrac12 \left(\dfrac 43 + \dfrac 9{16}\right)= \dfrac {91}{96}.$ Il vient:
  
  $p_k = \displaystyle \Pr\Big[\bigcup _{n=1}^{+\infty} (S_n+k =0)\Big]\leqslant \sum_{n=1}^{+\infty} \Pr [S_n +k =0] = \sum_{n=1}^{+\infty} \Pr \left[\left(\frac34\right ) ^{S_n+k} =1\right]$
  
  $p_k\leqslant \displaystyle \sum_{n=1}^{+\infty} \mathbb E \left(\left (\frac 34 \right)^{S_n+k}\right) = \left(\frac 34 \right)^k\:\: \sum _{n=1}^{+\infty} a^n.$ 
  
  On déduit: $\displaystyle \lim _{k\to + \infty} p_k =0,\:\: r=0,\:$ puis avec $p_0 =1, \:\:\:\: s=1, \quad p_1 = \alpha.$
  
  $$ \displaystyle \boxed {\Pr\Big[\bigcap_{n=1}^{+\infty} (Y_n \neq 0) \Big] =1-\alpha = \dfrac {3 -\sqrt 5}2}. $$
  
 
  

 
 
 

\section{Solution du deuxième exercice}
 
La réponse est : les applications de la forme $f: \displaystyle M\mapsto \phi(\mbox{Rg}(M))$ où $\displaystyle \phi : \{0,\ldots,n\}\rightarrow \mathbb{R}$ est croissante ($\phi$ est une $n$-liste croissante) sont les seules qui conviennent.

\begin{enumerate}
\item Soit $P\in \mbox{GL}_{n}(\mathbb{R}).$

On a  alors 
\begin{align*}
f(I_{n})=f(P.P^{-1}) & \leq f(P)\wedge f(P^{-1})\\
& \leq f(P)\\
\mbox{ et, } f(P)=f(P.I_{n})& \leq f(P)\wedge f(I_{n})\\
& \leq f(I_{n}).
\end{align*}

Ainsi, on a prouvé : $\displaystyle f(P)=f(I_{n}).$

\item Considérons $M$ une matrice de rang $r\in \{0,\ldots,n\}.$

On sait alors qu'il existe deux matrices inversibles $P$ et $Q$ telles que  $$M=P\left( \begin{array}{ll}
I_{r} & O_{r,n-r}\\
O_{n-r,r} & O_{n-r,n-r}
\end{array}
\right) Q:=PM_{r}Q \mbox{ i.e. } M_{r}=P^{-1}MQ^{-1}.$$

Donc, par la relation précédente, il vient : 
\begin{align*}
f(M) & \leq f(M_{r})\\
\mbox{ et, } f(M_{r}) & \leq f(M).
\end{align*}

Ainsi, on a prouvé : $\displaystyle f(M)=f(M_{r}).$

Et, $f$ est constante sur l'ensemble des matrices de rang fixé.

\item Considérons $r,r'\in \{0,\ldots,n\}$ tels que $r'\geq r.$

On a alors : $$f(M_{r})=f(M_{r}M_{r'})\leq f(M_{r'}).$$
Ainsi, $f$ est croissante en le rang de son argument.
\item On a alors montré que $f: \displaystyle M\mapsto \phi(\mbox{Rg}(M))$ où $\displaystyle \phi : \{0,\ldots,n\}\rightarrow \mathbb{R}$ est croissante.\\ 

La réciproque est immédiate car, on a toujours $\displaystyle \mbox{Rg}(XY)\leq \mbox{Rg}(X)\wedge\mbox{Rg}(Y).$ 

La croissance de $\phi$ permet alors de conclure.
\end{enumerate}


