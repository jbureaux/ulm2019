\chapter{Planche 36}

\section{Sujet}

\paragraph{Exercice}

\section{Solution du premier exercice}
Notons:$\quad\forall n \in \mathbb N  ,\quad X_n := (Y_{n+1},Y_{n+2}),\quad f,g: \mathbb N^2 \to \mathbb N^2 \quad$
 
 $ f:(a,b)\mapsto (b,a+b), \quad g:(a,b) \mapsto (b,|a-b|)\quad.$
  
 
  Soit $V$ l'ensemble des valeurs, autres que $(0,1)$ , que peuvent prendre les $X_n.\quad$ Alors: $\forall (a,b) \in V,\: a\wedge b =1, \quad \exists !\:\: k \in \mathbb N \:\:$ tel que $g^k (a,b) =(1,0).$
 
  Notons $k= \varphi (a,b)$ et $Z_n = \varphi (X_n)$
  La suite $(Z_n) _{n\in \mathbb N} $ est une chaîne de Markov dont l'ensemble des états est $\mathbb N$, telle que pour tout $(i,j)$ dans $\mathbb N^*\times \mathbb N$, la probabilité de passage de l'état $i$ à l'état $j$ est $p_{i,j} = \left\{ \begin{array} {cl} \frac12& \text{si}\: j=i+2 \: \:\text{ou}\:\: j=i-1. \\ 0 &\text{sinon}. \end{array} \right.$
  
  Pour tout entier naturel $k$, notons enfin $p_k$ la probabilité d'atteindre l'état $0$ à partir de l'état $k$. 
  Dans ce contexte, la probabilité $P$ que le jeu s'arrête est $P = p_1.$
  On a; $\quad p_0 =1,\quad \forall k\in \mathbb N^*, \:\: p_k = \dfrac 12 ( p_{k-1} + p_{k+2})$.
 
  Ainsi la suite $(p_k)$ vérifie une récurrence linéaire dont le polynôme caractéristique est $(X-1)(X-\alpha)(X- \beta)$ avec $\alpha =\dfrac {-1+\sqrt5}2,\:\: \beta = \dfrac{-1-\sqrt 5}2.$
  
  Il existe des réels $r,s,t$ tels que $\forall k \in \mathbb N, \quad p_k =r+s\alpha ^k+t \beta^k.$
  Le fait que $|\beta| >0 $ et le caractère borné de la suite $(p_k)$ impliquent   que $t=0.$
  
  
 
  

 
 
 

\section{Solution du deuxième exercice}
 
La réponse est : les applications de la forme $f: \displaystyle M\mapsto \phi(\mbox{Rg}(M))$ où $\displaystyle \phi : \{0,\ldots,n\}\rightarrow \mathbb{R}$ est croissante ($\phi$ est une $n$-liste croissante) sont les seules qui conviennent.

\begin{enumerate}
\item Soit $P\in \mbox{GL}_{n}(\mathbb{R}).$

On a  alors 
\begin{align*}
f(I_{n})=f(P.P^{-1}) & \leq f(P)\wedge f(P^{-1})\\
& \leq f(P)\\
\mbox{ et, } f(P)=f(P.I_{n})& \leq f(P)\wedge f(I_{n})\\
& \leq f(I_{n}).
\end{align*}

Ainsi, on a prouvé : $\displaystyle f(P)=f(I_{n}).$

\item Considérons $M$ une matrice de rang $r\in \{0,\ldots,n\}.$

On sait alors qu'il existe deux matrices inversibles $P$ et $Q$ telles que  $$M=P\left( \begin{array}{ll}
I_{r} & O_{r,n-r}\\
O_{n-r,r} & O_{n-r,n-r}
\end{array}
\right) Q:=PM_{r}Q \mbox{ i.e. } M_{r}=P^{-1}MQ^{-1}.$$

Donc, par la relation précédente, il vient : 
\begin{align*}
f(M) & \leq f(M_{r})\\
\mbox{ et, } f(M_{r}) & \leq f(M).
\end{align*}

Ainsi, on a prouvé : $\displaystyle f(M)=f(M_{r}).$

Et, $f$ est constante sur l'ensemble des matrices de rang fixé.

\item Considérons $r,r'\in \{0,\ldots,n\}$ tels que $r'\geq r.$

On a alors : $$f(M_{r})=f(M_{r}M_{r'})\leq f(M_{r'}).$$

Ainsi, $f$ est croissante en le rang de son argument.
\item On a alors montré que $f: \displaystyle M\mapsto \phi(\mbox{Rg}(M))$ où $\displaystyle \phi : \{0,\ldots,n\}\rightarrow \mathbb{R}$ est croissante.\\ 

La réciproque est immédiate car, on a toujours $\displaystyle \mbox{Rg}(XY)\leq \mbox{Rg}(X)\wedge\mbox{Rg}(Y).$ 

La croissance de $\phi$ permet alors de conclure.
\end{enumerate}


