\chapter*{Introduction}
 
Ce recueil n'existerait pas sans l'initiative remarquable du jury de l'épreuve orale « Maths Ulm » en 2019, composé d'Omid Amini et Igor Kortchemski, qui ont rendu public l'ensemble des exercices posés cette année-là.
Le document est disponible à l'adresse suivante :
\begin{center}
    \url{https://www.ens.fr/sites/default/files/2019_mathsulm_sujets-1.pdf}
\end{center}
Outre les sujets, il comporte de nombreuses indications et questions additionnelles qui n'ont pas systématiquement été reproduites ici.

\bigbreak

Les solutions proposées dans ce recueil n'ont rien d'officiel. Elles ont été rédigées par des membres du forum \href{http://www.les-mathematiques.net/phorum/}{les-mathematiques.net} de manière individuelle et totalement indépendante du jury du concours. Nous souhaitons toutefois qu'elles puissent participer favorablement au travail des étudiants et professeurs, dans l'esprit d'ouverture initié par le jury. Nous espérons aussi que ce recueil intéressera plus largement les amateurs de beaux problèmes mathématiques.

\bigbreak

Autant que possible, nous avons cherché à respecter le contenu et les notations des programmes des classes MPSI et MP. Nous avons rarement dérogé à cette règle, lorsque la clarté de la présentation nous semblait le justifier, et généralement en proposant des solutions alternatives

\bigbreak

Nous prions enfin le lecteur exigeant de bien vouloir pardonner les coquilles et autres défauts de ce recueil qui se présente aujourd'hui sous une forme encore incomplète et certainement imparfaite. 

\bigbreak

\noindent
Toute participation, correction, suggestion, etc. sera la bienvenue.
Les auteurs peuvent être contactés librement via  
\href{http://www.les-mathematiques.net/phorum/}{les-mathematiques.net}  :
\begin{center}
\href{http://www.les-mathematiques.net/phorum/read.php?4,1841908}{Forums > Analyse > Oral ENS Ulm, 36 planches}
\end{center}
