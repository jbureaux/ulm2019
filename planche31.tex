\chapter{Somme d'éléments du tore}

\section{Sujet}

\paragraph{Exercice}
Soit $m \geqslant 1$. Trouver tous les nombres complexes $a_1,\dots,a_m$ de module égal à $1$ tels que
\[
\sum_{j=1}^m a_j^n
\]
a une limite quand $n \to \infty$.


\section{Solution de l'exercice}

On procède par récurrence sur $m\geq 1.$
 
Pour le cas $m=1,$ on note $a_{1}=e^{i\theta}.$

Par hypothèse, on a $$\vert e^{i\theta}-1\vert =\vert e^{in\theta}-e^{i(n+1)\theta}\vert \longrightarrow_{n\rightarrow +\infty} 0.$$ On trouve alors $\theta=0\mbox{ }[2\pi]$ i.e. $a_{1}=1.$\\


Supposons que la suite $\displaystyle\left(A_{n}:=\sum_{k=1}^{m+1}a_{k}^{n}\right)_{n\geq 0}$ (où chacun des $a_{k}\in\mathbb{T}$) converge pour un certain $m\geq 1.$

Cette suite converge nécessairement vers $m+1.$ Montrons donc que $m+1$ est une valeur d'adhérence de cette suite.

Quitte à extraire en cascade et à numéroter de nouveau les termes de la suite, il existe une extraction $(\phi(n))$ telle que pour tout $k,$ $(a_{k}^{\phi(n)})$ converge (disons vers $l_{k}\in\mathbb{T}$) et telle que la suite $(\phi(n+1)-\phi(n))$ est strictement croissante.

Ainsi, $$A_{\phi_{n+1}-\phi(n)}\longrightarrow_{n\rightarrow +\infty} \sum_{k=1}^{m+1}l_{k}\overline{l_{k}}=\sum_{k=1}^{m+1}1=(m+1).$$ 
Prenons une valeur d'adhérence de la suite $(a_{1}^{n})$ que l'on note $l_{1}\in\mathbb{T}.$

Quitte à extraire en cascade, il existe une extraction $(\phi(n))$ telle que pour tout $k,$ $(a_{k}^{\phi(n)})$ converge (disons vers $l_{k}\in\mathbb{T}$).

On a alors en passant à la limite : $\displaystyle \sum_{k=1}^{m+1}l_{k}=m+1.$

Mais alors, par le cas d'égalité de l'inégalité triangulaire, on obtient pour tout $k\in\{1,\ldots,n\},$ $l_{k}=l_{1}$ (comme tous les nombres en jeu sont de module $1$) et donc $l_{1}=1.$

Ainsi, la suite $(a_{1}^{n})$  est convergente (vers $1$) et le premier pas de la récurrence nous indique que $a_{1}=1.$

On conclut par récurrence car la suite $\displaystyle \left(\sum_{k=2}^{m+1}a_{k}^{n}\right)_{n\geq 0}$ est convergente (vers $m$).

Ainsi, $\displaystyle \forall k\in \{1,\ldots,m+1\},\mbox{ } a_{k}=1.$