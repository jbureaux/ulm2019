\chapter{Coefficients du polynôme caractéristique}

\section{Sujet}

\paragraph{Exercice}

Soit $A \in \mathscr M_n(\mathbb R)$ et notons
\[
P(X) := \det(X\mathrm I_n - A) = X^n + c_1 X^{n-1} + c_2 X^{n-2} + \cdots + c_{n-1} X + c_n
\]
sont polynôme caractérisitque.
\begin{enumerate}[label = \arabic*)]
    \item En notant $\lambda_1,\dots,\lambda_n$ les racines de $P$, calculer $\sum_{k=1}^n \frac{P(X)}{X-\lambda_k}$ de deux manières différentes.
    \item Montrer que pour tout $1 \leqslant k \leqslant n$,
    \newcommand{\tr}[1]{\operatorname{tr}(#1)}
    \[
    c_k = \frac{(-1)^k}{k!}\begin{vmatrix}
         \tr{A} & 1 & 0 & 0 & \cdots & 0  \\
         \tr{A^2} & \tr{A} & 2 & 0 & \cdots & 0\\ 
         \tr{A^3} & \tr{A^2} & \tr{A} & 3 & \ddots & 0\\
         \vdots & \vdots & \vdots & \ddots & \ddots & 0\\
         \tr{A^{k-1}} & \tr{A^{k-2}} & \cdots & \cdots & \tr{A} & k-1\\
         \tr{A^k} & \tr{A^{k-1}} & \cdots & \cdots & \tr{A^2} & \tr{A}
    \end{vmatrix}.
    \]
\end{enumerate}

\paragraph{Deuxième exercice}

Trouver toutes les fonctions $f : \mathbb R_+ \to \mathbb R$ dérivables telles que $f(1)=1$ et $f(x)f(y) \leq f(xy)$ pour tous $x,y \geqslant 0$.

\section{Solution du premier exercice} % Siméon

(en cours)

\section{Solution du deuxième exercice} % BobbyJoe

Soit $y>0$
Comme $f$ est dérivable, on a pour $x$ proche de $1$ : 
$$f(x)=f(1)+(x-1)f'(1)+o(x-1)=1+(x-1)f'(1)+o(x-1)$$ mais aussi $$ f(xy)=f(y)+(x-1)yf'(y)+o(x-1).$$
Ainsi, on obtient : $$f(x)f(y)-f(xy)=(x-1)\left( f'(1)f(y)-yf'(y) \right)+o(x-1)\leq 0.$$
En divisant par $(x-1)$ (en traitant les cas $x>1$ et $x<1$) et en faisant tendre $x$ vers $1$, on obtient : $$yf'(y)=f'(1)f(y).$$
Il vient alors pour tout $y>0,$ $\displaystyle f(y)=y^{f'(1)}$ (car $f(1)=1$).
Cependant, $f$ doit être dérivable sur $\mathbb{R}^{+}$ donc $f'(1)\geq 1.$

Il est facile de vérifier que les fonctions du type $ f: x\mapsto x^{\lambda}$ où $\lambda\geq 1$ sont dérivables sur $\mathbb{R}^{+}$ et satisfont l'inéquation fonctionnelle.