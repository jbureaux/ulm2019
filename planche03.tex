\chapter{Coefficients du polynôme caractéristique}

\section{Sujet}

\paragraph{Exercice}

Soit $A \in \mathscr M_n(\mathbb R)$ et notons
\[
P(X) := \det(X\mathrm I_n - A) = X^n + c_1 X^{n-1} + c_2 X^{n-2} + \cdots + c_{n-1} X + c_n
\]
son polynôme caractéristique.
\begin{enumerate}[label = \arabic*)]
    \item En notant $\lambda_1,\dots,\lambda_n$ les racines de $P$, calculer $\sum_{k=1}^n \frac{P(X)}{X-\lambda_k}$ de deux manières différentes.
    \item Montrer que pour tout $1 \leqslant k \leqslant n$,
    \newcommand{\tr}[1]{\operatorname{tr}(#1)}
    \[
    c_k = \frac{(-1)^k}{k!}\begin{vmatrix}
         \tr{A} & 1 & 0 & 0 & \cdots & 0  \\
         \tr{A^2} & \tr{A} & 2 & 0 & \cdots & 0\\ 
         \tr{A^3} & \tr{A^2} & \tr{A} & 3 & \ddots & 0\\
         \vdots & \vdots & \vdots & \ddots & \ddots & 0\\
         \tr{A^{k-1}} & \tr{A^{k-2}} & \cdots & \cdots & \tr{A} & k-1\\
         \tr{A^k} & \tr{A^{k-1}} & \cdots & \cdots & \tr{A^2} & \tr{A}
    \end{vmatrix}.
    \]
\end{enumerate}

\paragraph{Deuxième exercice}

Trouver toutes les fonctions $f : \mathbb R_+ \to \mathbb R$ dérivables telles que $f(1)=1$ et $f(x)f(y) \leq f(xy)$ pour tous $x,y \geqslant 0$.

\section{Solution du premier exercice} % ev

\begin{enumerate}
 
\item 
 
On a la \og décomposition en éléments simples \fg{} \( \dfrac{P'(X)}{P(X)} = \displaystyle\sum_{k=1}^n \dfrac{1}{X-z_k} \) qui est une vraie décomposition en éléments simples lorsque les racines \( (z_k)_{1\leqslant k \leqslant n} \) sont simples.
On en déduit que \( \displaystyle\sum_{k=1}^n \dfrac{P(x)}{x-z_k} = P'(x) \).
 
D'autre part, pour tout entier \( m \in [\![0,n]\!], \; z_k^m \) est valeur propre de \( A^m \) et \( S_m = \sum\limits_{k=1}^n z_k^m = \operatorname{tr}(A^m) \).
On écrit alors pour \( \vert x \vert > \vert z_k \vert \) (donc \( x \neq 0\)),
\begin{align*}
\dfrac{P(x)}{x-z_k} &= \dfrac{P(x)}{x\left(1-\frac{z_k}{x}\right) }\\
&= \dfrac{P(x)}{x} \sum_{m=0}^{+\infty}\dfrac{z_k^m}{x^m}.
\end{align*}
 
En sommant pour \( k \) allant de \( 1 \) à \( n \), on trouve
 
\[ \sum_{k=1}^n \dfrac{P(x)}{x-z_k} = P(x) \sum_{m=0}^{+\infty}\dfrac{S_m}{x^{m+1}}. \]
 
\item 
 
En posant \( c_0 = 1 \), l'égalité précédente s'écrit \( P'(x) = P(x) \sum\limits_{m=0}^{+\infty}\dfrac{S_m}{x^{m+1}} \) ou\\
 
\( \sum\limits_{k=0}^{n-1} (n-k)c_kx^{n-k-1}  = \left( \sum\limits_{k=0}^{n} c_kx^{n-k} \right) \left( \sum\limits_{m=0}^{+\infty}\dfrac{S_m}{x^{m+1}} \right) \) et ce pour tout \( x > \vert z_k \vert \).
 
Il appert donc que le membre de droite est un polynôme malgré un aspect manifestement rébarbatif. Autrement dit les coefficients des \( x^{-k} \) sont nuls pour \( k > 0 \). On se souvient que \( S_0 = n \).
 
Soit \( k \in [\![1,n]\!] \).
 
\[ \begin{array}{rrrrrrrrrrrrrr}
\text{coeff. de}&x^{n-2}&:&(n-1)c_1&=&S_1c_0&+&S_0c_1& &      & & & &
\\
\text{coeff. de}&x^{n-3}&:&(n-2)c_2&=&S_2c_0&+&S_1c_1&+&S_0c_2& & & &
\\
                     &       & & & \vdots &      & &      & &      & & & &
\\
\text{coeff. de}&x^{n-k}&:&(n-k+1)c_{k-1}&=&S_{k-1}c_0&+&S_{k-2}c_1&+&\ldots&+&S_0c_{k-1}& &
\\
\text{coeff. de}&x^{n-k-1}&:&(n-k)c_{k}&=&S_{k}c_0&+&S_{k-1}c_1&+&\ldots&+&S_1c_{k-1}&+&S_0c_{k}
\end{array} \]
 
On soustrait le membre de gauche dans le membre de droite sauf pour la dernière ligne, où l'on ne soustrait que \( nc_k \) pour obtenir les \( k \) lignes.
 
\[ \begin{array}{rrrrrrrrr}
0&=&S_1c_0&+&c_1& &      & & 
\\
0&=&S_2c_0&+&S_1c_1&+&2c_2& & 
\\
& \vdots  &      & &      & &      & & 
\\
0&=&S_{k-1}c_0&+&S_{k-2}c_1&+&\ldots&+&(k-1)c_{k-1}
\\
-kc_{k}&=&S_{k}c_0&+&S_{k-1}c_1&+&\ldots&+&S_1c_{k-1}
\end{array} \tag{$\star$}\]
 
On considère pour \( k \in [\![1,n]\!] \) la proposition \( \mathcal P_k \) :
 
\[
\forall p \in [\![1,k]\!] \,, \quad
c_{p}=\frac{(-1)^{p}}{p !}\left|\begin{array}{ccccc}
S_1 & {1} & {0} & {\cdots} & {0} \\
S_2 & S_1 & {2} & {\ddots} & {\vdots} \\
{\vdots} & {\ddots} & {\ddots} & {\ddots} & {0} \\
S_{p-1} & {} & {\ddots} & {\ddots} & {p-1} \\
S_p & S_{p-1} & {\cdots} & S_2 & S_1
\end{array}\right|. \]
 
La proposition \( \mathcal P_1 \) est claire.
 

 
Soit \( k \in [\![2,n]\!] \) pour lequel on a \( \mathcal P_{k-1} \). 
 

 
\( \bullet \) Si on a
 
\[ A_k = \left(\begin{array}{ccccc}
S_1 & {1} & {0} & {\cdots} & {0} \\
S_2 & S_1 & {2} & {\ddots} & {\vdots} \\
{\vdots} & {\ddots} & {\ddots} & {\ddots} & {0} \\
S_{k-1} & {} & {\ddots} & {\ddots} & {k-1} \\
S_k & S_{k-1} & {\cdots} & S_2 & S_1
\end{array}\right)
\quad\text{et}\quad
\Delta_k = \det(A_k) \neq 0, \]
 
alors le système \( (\star) \) d'inconnue \( (c_0,c_1, \ldots, c_{k-1}) \in \mathbb C^k \) est de Cramér.\\ 
 
En particulier on a la formule de Cramér 
\( c_0 = \dfrac{N_k}{\Delta_k}, \) avec 
 
\[ N_k = \left|\begin{array}{ccccc}
0 & {1} & {0} & {\cdots} & {0} \\
0 & S_1 & {2} & {\ddots} & {\vdots} \\
{\vdots} & {\ddots} & {\ddots} & {\ddots} & {0} \\
0 & {} & {\ddots} & {\ddots} & {k-1} \\
-kc_k & S_{k-1} & {\cdots} & S_2 & S_1
\end{array}\right| = (-1)^{k-1}(-kc_k)\left|\begin{array}{cccc}
 {1} & {0} & {\cdots} & {0} \\
 S_1 & {2} & {\ddots} & {\vdots} \\
{\vdots}  & {\ddots} & {\ddots} & {0} \\
 S_{k-2} & {\ldots} & S_1 & {k-1} 
\end{array}\right|.  \]
 
Donc \( N_k = (-1)^{k}kc_k\times(k-1)! = (-1)^{k}k!c_k \) et on a bien 
 
\[ c_k = (-1)^{k}c_0\dfrac{\Delta_k}{k!} = (-1)^{k}\dfrac{\Delta_k}{k!} \]
 
puisque \( c_0 = 1 \).\\
 

 
\( \bullet \) Si on a \( \Delta_k = 0 \) alors les \( k-1 \) premières lignes de \( A_k \) sont indépendantes donc la dernière ligne est combinaison linéaire des \( k-1 \) premières~: \( L_k = \sum\limits_{p=1}^{k-1} \lambda_p \). 
 
Soit \( f_p \) la forme linéaire définie par la \( p \)-ième ligne de \( A_k = (a_{p,\ell})_{1\leqslant p,\ell \leqslant k} \)~:
\[ f_p~;~(x_0,x_1, \ldots, x_{k-1}) \longmapsto  a_{p,1}x_0 + a_{p,2}x_1 + \ldots + a_{p,k}x_{k-1}. \]
 
Comme le membre de gauche des \( k-1 \) premières lignes de \( A_k \) est nul, on a
\( (c_0,c_1, \ldots, c_{k-1}) \in \ker f_p \) pour \( 1\leqslant p < k \), 
donc \( (c_0,c_1, \ldots, c_{k-1}) \in \ker f_k \). Autrement dit 
 
\[ -kc_k = f_k(c_0,c_1, \ldots, c_{k-1}) = 0. \]
 
C'est bien dire que \( c_k = 0 = (-1)^{k}c_0\dfrac{\Delta_k}{k!} \).
 

 
Dans les deux cas, on a bien \( \mathcal P_{k} \).
 

 
En se souvenant que \( S_p = \operatorname{tr}\left(A^{p}\right) \), on a ainsi démontré par récurrence finie que 
 
\[ \forall k \in[\![1,n]\!], \; c_{k}=\frac{(-1)^{k}}{k !}\left|\begin{array}{ccccc}
{\operatorname{tr}(A)} & {1} & {0} & {\cdots} & {0} \\
{\operatorname{tr}\left(A^{2}\right)} & {\operatorname{tr}(A)} & {2} & {\ddots} & {\vdots} \\
{\vdots} & {\ddots} & {\ddots} & {\ddots} & {0} \\
{\operatorname{tr}\left(A^{k-1}\right)} & {} & {\ddots} & {\ddots} & {k-1} \\
{\operatorname{tr}\left(A^{k}\right)} & {\operatorname{tr}\left(A^{k-1}\right)} & {\cdots} & {\operatorname{tr}\left(A^{2}\right)} & {\operatorname{tr}(A)}
\end{array}\right|. \]
 
\end{enumerate}


\section{Solution du deuxième exercice} % BobbyJoe

Soit $y>0$
Comme $f$ est dérivable, on a pour $x$ proche de $1$ : 
$$f(x)=f(1)+(x-1)f'(1)+o(x-1)=1+(x-1)f'(1)+o(x-1)$$ mais aussi $$ f(xy)=f(y)+(x-1)yf'(y)+o(x-1).$$
Ainsi, on obtient : $$f(x)f(y)-f(xy)=(x-1)\left( f'(1)f(y)-yf'(y) \right)+o(x-1)\leq 0.$$
En divisant par $(x-1)$ (en traitant les cas $x>1$ et $x<1$) et en faisant tendre $x$ vers $1$, on obtient : $$yf'(y)=f'(1)f(y).$$
Il vient alors pour tout $y>0,$ $\displaystyle f(y)=y^{f'(1)}$ (car $f(1)=1$).
Cependant, $f$ doit être dérivable sur $\mathbb{R}^{+}$ donc $f'(1)\geq 1.$

Il est facile de vérifier que les fonctions du type $ f: x\mapsto x^{\lambda}$ où $\lambda\geq 1$ sont dérivables sur $\mathbb{R}^{+}$ et satisfont l'inéquation fonctionnelle.