\chapter{Planche 6}

\section{Sujet}

\paragraph{Exercice}

\paragraph{Deuxième exercice}
Soit $G$ un groupe. A-t-on $G$ fini si et seulement si le nombre de sous-groupes de $G$ est fini ?

\section{Solution du deuxième exercice}

Montrons que les deux conditions sont bien équivalentes. On remarque déjà que si 
$G$ est fini, alors tout sous-groupe de $G$ appartient à l'ensemble $\mathscr P(G)$ des parties de $G$, qui est bien un ensemble fini.

Réciproquement, supposons que le nombre de sous-groupes de $G$ soit fini mais que $G$ soit infini et cherchons une contradiction.
Par principe des tiroirs, on dispose d'un sous-groupe monogène $H$ et d'une partie infinie $A$ de $G$ telle que tout élément de $A$ engendre $H$.
Mais alors $H$ est infini donc isomorphe à $(\mathbb Z,+)$. Ainsi, $H$ admet une infinité de sous-groupes et donc $G$ aussi.
