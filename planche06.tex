\chapter{Variante d'un théorème de Bernstein}

\section{Sujet}

\paragraph{Premier exercice}
Soit $f : [0, 1] \rightarrow  \mathbb{R}$ une fonction $\mathscr{C}^{\infty }$ non constante, telle que pour tout $n > 0$ et $x \in [0, 1]$ on a $f^{(n)}(x) \geqslant  0$. Montrer que $f$ ne peut s'annuler qu'au plus une fois. Montrer que $f$ est développable en série entière autour de tout point de $]0, 1[$ \emph{(cette deuxième question a parfois été posée une fois la première question résolue)}.

\paragraph{Deuxième exercice}
Soit $G$ un groupe. A-t-on $G$ fini si et seulement si le nombre de sous-groupes de $G$ est fini ?

\section{Solution du premier exercice} % Calli

Supposons que $f$ admette au moins deux zéros et cherchons une contradiction. Dans ce cas, le réel $a = \sup \{x\in  [0,1] \mid f(x) = 0 \}$ est strictement positif et vérifie $f(a) = 0$ par continuité. Puisque $f$ est positive et croissante, on en déduit qu'elle est nulle sur l'intervalle $[0,a]$, d'intérieur non vide, et donc : $\forall n\in \mathbb{N},\  f^{(n)} (a) = 0$. Soit maintenant $x \in \left]a,1\right[$. Alors d'après la formule de Taylor avec reste intégral appliquée en $a$ :
\[
\forall  n\in \mathbb{N},\qquad
f(x) = \int_a^x \frac{(x-t)^n}{n!} f^{(n+1)}(t) \,\mathrm dt.
\]
En remarquant que $t\mapsto\frac{x-t}{1-t}$ est positive et décroissante sur  $[0,x]$, on en déduit :
\[
f(x) \leqslant x^n \int _{a} ^{x} \frac{(1-t)^{n} }{n!} f^{(n+1)}(t) \,\mathrm{d}t
\leqslant 
x^n \int _{a} ^{1} \frac{(1-t)^{n} }{n!} f^{(n+1)}(t) \,\mathrm{d}t,
\]
d'où $0 \leqslant f(x) \leqslant x^n f(1)$ et finalement $f(x) = 0$ par passage à la limite. Ceci étant vrai quel que soit $x \in \left]a,1\right[$, la fonction $f$ est nulle sur $[0,1[$ et donc constante sur $[0,1]$ par continuité, ce qui est contraire aux hypothèses.\\

En réalité, nous venons de montrer que $f$ est développable en série entière autour de $a$ et que cette série est nulle. Montrons plus généralement que si $f$ satisfait les hypothèses de l'énoncé, alors elle est développable en série entière en tout point de $\left]0,1\right[$ (c'est l'objet de la deuxième partie de la question).

Fixons $a\in \,]0,1[$ et notons $R_{n} (x)$ le reste intégral de Taylor à l'ordre $n$ centré en $a$ et évalué en $x$. On a de même que précédemment : 
\[
\forall x\in [a,1[,\quad
0 \;\leqslant \; R_{n} (x) 
\;\leqslant \;
 x^{n} R_{n} (1)
 \;\leqslant \;
 x^{n} f(1) \;
 \xrightarrow[n\rightarrow\infty]{} 0,
\]
d'où : $f(x)=\displaystyle \sum _{n=0} ^{\infty } \frac{f^{(n)}(a)}{n!} (x-a)^{n}$ (et c'est même vrai en $x=1$ car cette série entière est à termes positifs).
%
De plus : $\forall x\in \left]\max (0,a-(1-a)),a\right[,$
\begin{eqnarray*}
|R_{n} (x)| &=& \left| \int _{a} ^{x} \frac{(x-t)^{n} }{n!} f^{(n+1)}(t) \,\mathrm{d}t \right|\\
&=& \int _{x} ^{a} \frac{(t-x)^{n} }{n!} f^{(n+1)}(t) \,\mathrm{d}t\\
&\leqslant & \int _{x} ^{a} \frac{(t-x)^{n} }{n!} f^{(n+1)}(a) \,\mathrm{d}t\\
&=& \frac{f^{(n+1)}(a)}{n!} \, |x-a|^{n} \cdot  |x-a| \\
&\text{\makebox[0mm]{$\xrightarrow[n\rightarrow\infty]{}$}} & \ \ 0
\end{eqnarray*}
car, la série entière $\displaystyle \sum _{k=0} ^{\infty } \frac{f^{(n+1)}(a)}{n!} (x-a)^{n}$ -- dérivée de $\displaystyle \sum _{n=0} ^{\infty } \frac{f^{(n)}(a)}{n!} (x-a)^{n}$ -- ayant un rayon de convergence supérieur ou égal à $1-a$, son terme général tend vers 0 en $x$. Ainsi $f$ est développable en série entière dans un rayon de $\min(1-a,a)$ autour de $a$.


\section{Solution du deuxième exercice} % Siméon

Montrons que les deux conditions sont bien équivalentes. On remarque déjà que si 
$G$ est fini, alors tout sous-groupe de $G$ appartient à l'ensemble $\mathscr P(G)$ des parties de $G$, qui est bien un ensemble fini.

Réciproquement, supposons que le nombre de sous-groupes de $G$ soit fini mais que $G$ soit infini et cherchons une contradiction. Considérons l'application qui associe à tout élément de $G$ son sous-groupe engendré.
Par principe des tiroirs, on dispose d'un sous-groupe $H$ et d'une partie infinie $A$ de $G$ telle que tout élément de $A$ engendre $H$.
Mais alors $H$ est monogène et infini, donc isomorphe à $(\mathbb Z,+)$. Or $(\mathbb Z,+)$ admet une infinité de sous-groupes (les $n\mathbb Z$ pour $n \in \mathbb N$) donc $H$ et $G$ aussi.
