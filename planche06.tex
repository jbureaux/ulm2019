\chapter{Planche 6}

\section{Sujet}


\paragraph{Premier exercice}
Soit $f : [0, 1] \rightarrow  \mathbb{R}$ une fonction $\mathcal{C}^{\infty }$ non constante, telle que pour tout $n > 0$ et $x \in [0, 1]$ on a $f^{(n)}(x) \geqslant  0$. Montrer que $f$ ne peut s'annuler qu'au plus une fois. Montrer que $f$ est développable en série entière autour de tout point de $]0, 1[$ \emph{(cette deuxième question a parfois été posée une fois la première question résolue)}.

\paragraph{Deuxième exercice}
Soit $G$ un groupe. A-t-on $G$ fini si et seulement si le nombre de sous-groupes de $G$ est fini ?

\section{Solution de l'exercice} % Calli

Supposons que $f$ admet au moins deux zéros et cherchons une contradiction. On pose $a = \sup \{x\in  [0,1] \mid f(x) = 0 \}$, qui est strictement positif. Par croissance et positivité de $f$, on a : $\forall x\in [0,a],\ 0\leqslant f(x)\leqslant f(a)=0$, donc $f|_{[0,a]} = 0$. D'où : $\forall n\in \mathbb{N}, f^{(n)} (a) = 0$. La formule de Taylor avec reste intégral appliquée en $a$ donne alors : $\forall  x \in  \left]a,1\right[, \forall  n\in \mathbb{N},$
\begin{eqnarray*}
0 \,=\, f(a) \,\leqslant \, f(x) &=& \int _{a} ^{x} \frac{(x-t)^{n} }{n!} f^{(n+1)}(t) \,\mathrm{d}t\\
&=& \int _{a} ^{x} \left( \frac{x-t}{1-t} \right)^{n} \frac{(1-t)^{n} }{n!} f^{(n+1)}(t) \,\mathrm{d}t\\
&\leqslant & x^{n} \int _{a} ^{x} \frac{(1-t)^{n} }{n!} f^{(n+1)}(t) \,\mathrm{d}t \ \ (*)\\
&\leqslant & x^{n} \int _{a} ^{1} \frac{(1-t)^{n} }{n!} f^{(n+1)}(t) \,\mathrm{d}t\\
&=& x^{n} f(1) \\
&\text{\makebox[0mm]{$\xrightarrow[n\rightarrow\infty]{}$}} & \ \ 0
\end{eqnarray*}
où $(*)$ provient de la croissance et de la positivité de $t\mapsto\frac{x-t}{1-t}$ sur $[0,x]$. Donc $f|_{[0,1[} =0$ et finalement $f=0$ par continuité en 1, ce qui conclut.

En réalité, nous venons de montrer que $f$ est développable en série entière autour de $a$ et que cette série est nulle. Montrons plus généralement que si $f$ satisfait les hypothèses de l'énoncé, alors elle est développable en série entière en tout point de $\left]0,1\right[$ (c'est l'objet de la deuxième partie de la question).

Fixons $a\in \,]0,1[$ et notons $R_{n} (x)$ le reste intégral de Taylor à l'ordre $n$ centré en $a$ et évalué en $x$. On a de même que précédemment : $\forall x\in [a,1[,$
\begin{eqnarray*}
0 \;\leqslant \; R_{n} (x) &=& \int _{a} ^{x} \left( \frac{x-t}{1-t} \right)^{n} \frac{(1-t)^{n} }{n!} f^{(n+1)}(t) \,\mathrm{d}t\\
&\leqslant & x^{n} \int _{a} ^{1} \frac{(1-t)^{n} }{n!} f^{(n+1)}(t) \,\mathrm{d}t\\
&=& x^{n} R_{n} (1)\\
&\leqslant & x^{n} \left(\sum_{k=0}^n \frac{f^{(k)}(a)}{k!} (1-a)^{k} +R_{n} (1)\right)\\
&=& x^{n} f(1) \\
&\text{\makebox[0mm]{$\xrightarrow[n\rightarrow\infty]{}$}} & \ \ 0
\end{eqnarray*}
Donc : $\forall x\in [a,1[,f(x)=\displaystyle \sum _{n=0} ^{\infty } \frac{f^{(n)}(a)}{n!} (x-a)^{n}$ (et c'est même vrai en $x=1$ car cette série entière est à termes positifs).

De plus : $\forall x\in \,]\max (0,a-(1-a)),a[,$
\begin{eqnarray*}
|R_{n} (x)| &=& \left| \int _{a} ^{x} \frac{(x-t)^{n} }{n!} f^{(n+1)}(t) \,\mathrm{d}t \right|\\
&=& \int _{x} ^{a} \frac{(t-x)^{n} }{n!} f^{(n+1)}(t) \,\mathrm{d}t\\
&\leqslant & \int _{x} ^{a} \frac{(t-x)^{n} }{n!} f^{(n+1)}(a) \,\mathrm{d}t\\
&=& \frac{f^{(n+1)}(a)}{n!} \, |x-a|^{n} \cdot  |x-a| \\
&\text{\makebox[0mm]{$\xrightarrow[n\rightarrow\infty]{}$}} & \ \ 0
\end{eqnarray*}
car, la série entière $\displaystyle \sum _{k=0} ^{\infty } \frac{f^{(n+1)}(a)}{n!} (x-a)^{n}$ -- dérivée de $\displaystyle \sum _{n=0} ^{\infty } \frac{f^{(n)}(a)}{n!} (x-a)^{n}$ -- ayant un rayon de convergence supérieur ou égal à $1-a$, son terme général tend vers 0 en $x$. Ainsi $f$ est développable en série entière dans un rayon de $1-a$ autour de $a$.


\section{Solution du deuxième exercice} % Siméon

Montrons que les deux conditions sont bien équivalentes. On remarque déjà que si 
$G$ est fini, alors tout sous-groupe de $G$ appartient à l'ensemble $\mathscr P(G)$ des parties de $G$, qui est bien un ensemble fini.

Réciproquement, supposons que le nombre de sous-groupes de $G$ soit fini mais que $G$ soit infini et cherchons une contradiction.
Par principe des tiroirs, on dispose d'un sous-groupe monogène $H$ et d'une partie infinie $A$ de $G$ telle que tout élément de $A$ engendre $H$.
Mais alors $H$ est infini donc isomorphe à $(\mathbb Z,+)$. Ainsi, $H$ admet une infinité de sous-groupes et donc $G$ aussi.
