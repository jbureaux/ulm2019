\chapter{Planche 11}

\section{Sujet}

\paragraph{Deuxième exercice}
Pour $n \geqslant 3$, on note $u_n$ la plus petite solution de l'équation $x = n \ln(x)$ pour $x \in \mathbb R_+^*$. Étudier $u_n$.

\section{Solution du deuxième exercice}

Pour tout $n \geqslant 3$, on a nécessairement $\ln(u_n) = u_n/n > 0$, donc $u_n > 1$.
Considérons la fonction $f : \left]1,e\right[ \to \left]e,+\infty\right[$ définie par $x \mapsto x/{\ln(x)}$.
C'est une bijection strictement décroissante de classe $\mathscr C^\infty$.
On en déduit que la suite $u$ est à valeurs dans $\left]1,e\right[$, strictement décroissante, de limite $1$, et vérifie :
\[
\forall n\geqslant 3,\quad u_n = f^{-1}(n).
\]

