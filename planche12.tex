\chapter{Planche 12}

\section{Sujet}

\paragraph{Exercice}
Soit $f : \mathbb R^n \to \mathbb R$ une fonction continue dont la restriction sur toute droite de $\mathbb R^n$ est monotone (c'est-à-dire, pour tous $u,v \in \mathbb R^n$, la fonction $t\mapsto f(tu+v)$ est monotone).
Montrer qu'il existe une forme linéaire $\phi : \mathbb R^n \to \mathbb R$ et une fonction monotone $h : \mathbb R \to \mathbb R$ telles que $f = h \circ \phi$.

\paragraph{Deuxième exercice}

\section{Solution de l'exercice}

Le résultat est clair si $f$ est constante sur toute droite.
On supposera donc qu'on dispose d'une droite $\mathcal D$ sur laquelle $f$ n'est pas constante.

Il existe alors un point $O \in \mathcal D$ tel que $O$ est l'unique antécédent de $f(O)$ sur $\mathcal D$. En effet, dans le cas contraire, le théorème des valeurs intermédiaires entraînerait l'existence d'une injection de l'intervalle non trivial $f(\mathcal D)$ dans $\mathbb Q$.
Quitte à ajouter une constante à $f$, on peut supposer que $f(O)=0$.

L'ensemble $\mathbb R^n \setminus\{O\}$ est connexe par arcs et son image par $f$ contient à la fois des réels positifs et des réels négatifs.
Par continuité de $f$, il existe donc un point $A_1 \in \mathbb R^n \setminus \mathcal D$ tel que $f(A_1) = 0$.
En raisonnant par récurrence sur la dimension, on obtient ainsi des points $A_1,A_2,\dots,A_{n-1}$ tels que $\{O,A_1,\dots,A_{n-1}\}$ engendre un hyperplan affine $\mathcal H$ de $\mathbb R^n$ et $f(A_i) = 0$ pour tout $i\in \{1,\dots,n-1\}$.

Montrons que $f$ est nulle sur $\mathcal H$. Compte tenu de l'hypothèse sur $f$, il suffit de prouver la nullité sur chacune des droites $(OA_i)$ puisque tout élément de $\mathcal H$ est barycentre d'éléments de ces droites.
Plaçons-nous dans le plan $\mathcal P_i$ engendré par $\mathcal D$ et $(OA_i)$, pour un $i \in \{1,\dots,n-1\}$ quelconque.
Le point $O$ sépare $\mathcal D$ en deux demi-droites sur lesquelles $f$ est respectivement strictement positive et strictement négative.
Puisque $f(A_i) = 0$, la condition de monotonie entraîne alors que la droite $(OA_i)$ sépare $\mathcal P_i$ en deux demi-plans ouverts sur lesquels $f$ est respectivement positive et négative (faire un dessin).
On en déduit la nullité de $f$ sur $(OA_i)$ par continuité.

Soit maintenant $M$ un point quelconque de $\mathcal D\setminus \{O\}$, et $\mathcal H_M$ l'hyperplan parallèle à $\mathcal H$ passant par $M$.
Toute droite passant par $M$ et n'appartenant pas à $\mathcal H_M$ intersecte $\mathcal H$.
Quitte à échanger les signes, on peut supposer que $f(M) > 0$.
La condition de monotonie entraîne alors que $f$ est à valeurs dans $[0;f(M)]$ entre les deux hyperplans (strictement), et à valeurs dans $\left[f(M);+\infty\right[$ sur le demi-espace ouvert délimité par $\mathcal H_M$ qui ne contient pas $\mathcal H$ (faire un dessin). Par continuité, on en déduit que $f$ est constante sur $\mathcal H_M$.

Considérons maintenant $D$ et $H$ les directions vectorielles respectives de $\mathcal D$ et $\mathcal H$.
Le projecteur sur $D$ parallèlement à $H$ est de la forme $\pi : x \mapsto \phi(x)u$ où $\phi$ est une forme linéaire de noyau $H$ et $u$ est un vecteur directeur de $D$.
D'après tout ce qui précède, on sait que $f = f \circ \pi$.
L'application $h : t \mapsto f(tu)$, qui est monotone par hypothèse sur $f$, vérifie donc bien l'égalité $f = h \circ \phi$.
