\chapter{Planche 35}

\section{Sujet}

\paragraph{Exercice}
Soient $(X_i)_{i\geqslant 1}$ des variables aléatoires i.i.d. telles que que $\mathbb P(X_i = 1) = \mathbb P(X_i = -1) = \frac12.$
On pose $S_n = X_1 + \cdots + X_n$.
Pour $\varepsilon > 0$, calculer
\[
\lim_{n\to\infty} \mathbb P\left(\left|\frac{S_n}n\right| > \varepsilon \mid S_{2n} = 0\right).
\]

\section{Solution de l'exercice}
Vérifions que cette limite est nulle.
Il suffit pour cela de montrer que, lorsque $n \to \infty$, la probabilité de l'évènement $(|S_n| \geq \varepsilon n)$ est négligeable par rapport à celle de l'évènement $(S_{2n} = 0)$, par lequel on conditionne. En effet, tous les évènements $A,B$ avec $\mathbb P(B) \neq 0$ vérifient l'inégalité grossière :
\[
\mathbb P(A \mid B) \leqslant \frac{\mathbb P(A)}{\mathbb P(B)}.
\]

La probabilité de $(S_{2n} = 0)$ se calcule explicitement par dénombrement ou bien en remarquant que $\frac12(S_{2n}+{2n})$ suit une loi binomiale de paramètres $2n$ et $\frac12$ :
\[
\mathbb P(S_{2n} = 0) = \frac1{2^{2n}}\binom{2n}{n} \underset{n\to\infty}{\sim} \frac 1{\sqrt{\pi n}},
\]
l'équivalent découlant de la formule de Stirling. Compte tenu des remarques précédentes, l'inégalité de Bienaymé-Tchebycheff suffit alors pour conclure : puisque $S_n$ est centrée et de variance $n$ (somme de variables indépendantes),
\[
 \mathbb P\left(|S_n| > \varepsilon n\right) \leqslant \frac{1}{\varepsilon^2 n}.
\]

\paragraph{Remarque}
Une question plus difficile à laquelle les examinateurs ont peut-être pensé serait le calcul de :
\[
\lim_{n\to\infty} \mathbb P\left(\left|\frac{S_n}{\sqrt n}\right| > \varepsilon \mid S_{2n} = 0\right).
\]

\section{Solution du deuxième exercice}

\begin{itemize}
\item Pour $c<0,$ la fonction $\displaystyle f: x\mapsto e^{x}$ satisfait les inéquations différentielles.
\item Pour $c=0,$ les fonctions $\displaystyle f_{\lambda} :x \mapsto e^{\lambda x}$ où $\lambda>1$ satisfont les inéquations différentielles.
\item Pour $c>0,$ il n'y a pas de fonction $\mathcal{C}^{2}$ qui vérifie les deux inéquations différentielles.
Par l'absurde, supposons qu'il existe une telle fonction $f$...
\end{itemize}

L'idée est la suivante, on a va voir que les inégalités forcent $f'$ et $f$ à être négatives dans un voisinage de $-\infty.$
Mais alors $f$ est décroissante dans ce voisinage de $-\infty$ et tend vers $+\infty$ en $-\infty,$ contredisant sa négativité!
\\

Par la méthode de la variation de la constante, on a pour tout $x\in\mathbb{R},$
\begin{align*}
f'(x) & =e^{x}\int_{0}^{x}\left(f''(t)-f'(t)\right)e^{-t}dt+f'(0)e^{x}\\
\mbox{ et } f(x) & =e^{x}\int_{0}^{x}\left(f'(t)-f(t)\right)e^{-t}dt+f(0)e^{x}.
\end{align*}

Il vient alors par croissance de l'intégrale et pour tout $x\leq 0,$ \begin{align*}
f'(x) & \leq ce^{x}\int_{x}^{0}-e^{-t}dt+f'(0)e^{x} = c(e^{x}-1)+f'(0)e^{x}\\
\mbox{ et } f(x)& \leq ce^{x}\int_{x}^{0}-e^{-t}dt+f(0)e^{x} = c(e^{x}-1)+f(0)e^{x}.
\end{align*}
En particulier, on a $\displaystyle \limsup_{x\rightarrow +\infty} f(x) \leq -c<0.$ 

\textbf{Remarque : } Une autre façon d'écrire ceci est de remarquer que le membre de droite tend vers $-c$ en $-\infty$ et donc $f'$ est négative dans un voisinage de $-\infty.$ Ainsi, $f$ est décroissante dans un voisinage de $-\infty$ et donc admet une limite en $-\infty$ qui est strictement négative en passant à la limite dans la deuxième inégalité.
\\

Finalement, en intégrant entre $x$ et $0$ la première inégalité (lorsque $x\leq 0$), il vient : 
\begin{align*}
f(0)-f(x) & \leq (c+f'(0))(1-e^{x})+cx\\
\mbox{ i.e. } f(x) & \geq f(0)-(c+f'(0))(1-e^{x})-cx.
\end{align*}

Par comparaison, on obtient $\displaystyle \lim_{x\rightarrow -\infty} f(x)=+\infty$ alors que $f$ est négative dans un voisinage de $-\infty,$ ce qui constitue la contradiction recherchée!
