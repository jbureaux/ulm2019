\chapter{Planche 35}

\section{Sujet}

\paragraph{Exercice}
Soient $(X_i)_{i\geqslant 1}$ des variables aléatoires i.i.d. telles que que $\mathbb P(X_i = 1) = \mathbb P(X_i = -1) = \frac12.$
On pose $S_n = X_1 + \cdots + X_n$.
Pour $\varepsilon > 0$, calculer
\[
\lim_{n\to\infty} \mathbb P\left(\left|\frac{S_n}n\right| > \varepsilon \mid S_{2n} = 0\right).
\]

\paragraph{Deuxième exercice}

\section{Solution de l'exercice}
Vérifions que cette limite est nulle.
Il suffit pour cela de montrer que, lorsque $n \to \infty$, la probabilité de l'évènement $(|S_n| \geq \varepsilon n)$ est négligeable par rapport à celle de l'évènement $(S_{2n} = 0)$, par lequel on conditionne. En effet, tous évènements $A,B$ avec $\mathbb P(B) \neq 0$ vérifient l'inégalité grossière :
\[
\mathbb P(A \mid B) \leqslant \frac{\mathbb P(A)}{\mathbb P(B)}.
\]

La probabilité de $(S_{2n} = 0)$ se calcule explicitement par dénombrement ou bien en remarquant que $\frac12(S_{2n}+{2n})$ suit une loi binomiale de paramètres $2n$ et $\frac12$ :
\[
\mathbb P(S_{2n} = 0) = \frac1{2^{2n}}\binom{2n}{n} \underset{n\to\infty}{\sim} \frac 1{\sqrt{\pi n}},
\]
l'équivalent découlant de la formule de Stirling. Compte tenu des remarques précédentes, l'inégalité de Bienaymé-Tchebycheff suffit alors pour conclure : puisque $S_n$ est centrée et de variance $n$ par indépendance,
\[
 \mathbb P\left(|S_n| > \varepsilon n\right) \leqslant \frac{1}{\varepsilon^2 n}.
\]

\paragraph{Remarque}
Une question plus difficile à laquelle les examinateurs ont peut-être pensé serait de calculer :
\[
\lim_{n\to\infty} \mathbb P\left(\left|\frac{S_n}{\sqrt n}\right| > \varepsilon \mid S_{2n} = 0\right).
\]
