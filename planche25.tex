\chapter{Planche 25}

\section{Solution de l'exercice}

On montre facilement par récurrence sur $k\in\mathbb{N}$ que pour tout $x\in[0,1],$ $$P_{k}(x);Q_{k}(x)\in[0,1].$$ 
On a également pour tout $k\geq 1,$ $$P_{k}(0)=1;Q_{k}(0)=0.$$

Soit $k\geq 0.$ Considérons $x\in[0,1].$ On a  alors en utilisant les relations de récurrence
$$P_{k+1}(x)=\left(1-x+\int_{0}^{x}Q_{k}(t)dt\right)=\left(1-\int_{0}^{x}\left(1-\int_{0}^{t}P_{k}(u)du\right)^{2}dt\right)^{2}.$$

On a ainsi en utilisant une identité remarquable (et comme les fonctions $P_{k}$ sont à valeurs dans $[0,1]$) : 
\begin{align*}
\vert P_{k+2}(x)-P_{k+1}(x)\vert  & = \left( 2-\int_{0}^{x}\left[ \left( 1-\int_{0}^{t}P_{k+1}(u)du \right)^{2}+\left( 1-\int_{0}^{t}P_{k+1}(u)du \right)^{2}\right]dt\right)\\
& \mbox{ }\mbox{ }\mbox{ }\mbox{ }\mbox{ }\times \left\vert\int_{0}^{x}\left[ \left( 1-\int_{0}^{t}P_{k+1}(u)du\right)^{2}-\left(1-\int_{0}^{t}P_{k+1}(u)du \right)^{2}\right]dt\right\vert\\
& \leq 2\int_{0}^{x}\left( 2-\int_{0}^{t}P_{k+1}(u)du-\int_{0}^{t}P_{k+1}(u)du\right)dt\times \int_{0}^{x}\left(\int_{0}^{t}\big\vert P_{k+1}(u)-P_{k}(u)\big\vert du\right)dt \\
& \mbox{ en appliquant l'inégalité triangulaire intégrale}\\
& \leq 4\int_{0}^{x}\left(\int_{0}^{t}\big\vert P_{k+1}(u)-P_{k}(u)\big\vert du\right)dt \mbox{ (*) }.
\end{align*}

De même, on a également en écrivant $\displaystyle x=\int_{0}^{x}1dt,$ 
\begin{align*}
\vert P_{k+1}(x)+Q_{k+1}(x)-1\vert & = \big\vert \int_{0}^{x}\left(P_{k}(t)+Q_{k}(t)-1\right)dt \big\vert \times \left(2-x+\int_{0}^{x}(Q_{k}(t)-P_{k}(t))dt\right)\\
& \leq \left(2-x+\int_{0}^{x}\big\vert Q_{k}(t)-P_{k}(t)\big\vert dt\right)\times \int_{0}^{x}\big\vert P_{k}(t)+Q_{k}(t)-1\big\vert dt\\
& \mbox{ en appliquant l'inégalité triangulaire intégrale}\\
& \leq 2\int_{0}^{x}\big\vert P_{k}(t)+Q_{k}(t)-1\big\vert dt \mbox{ (**)}.
\end{align*}

On montre alors (grâce aux formules $(*)$ et $(**)$) par récurrence sur $k\geq 0,$ $$\forall t\in[0,1] :\mbox{ } \vert P_{k+1}(t)-P_{k}(t)\vert\leq \frac{(2t)^{2k}}{(2k)!}  \mbox{ et } \vert P_{k}(t)+Q_{k}(t)-1\vert \leq \frac{(2t)^{k}}{k!}.$$

Les initialisations sont claires car $\displaystyle P_{0}=0=Q_{0} \mbox{ et } P_{1}(t)=(1-t)^{2}\leq 1.$

Enfin par $(*),$ on a en intégrant en cascade :
\begin{align*}
\vert P_{k+2}(x)-P_{k+1}(x)\vert  & \leq 4\int_{0}^{x}\left(\int_{0}^{t}\frac{(2u)^{2k}}{(2k)!}du\right)dt\\
& \leq \frac{4\times 2^{2k}}{(2k+1)!}\int_{0}^{x}t^{2k+1}dt\\
& \leq \frac{(2t)^{2k+2}}{(2k+2)!}.
\end{align*}

Et par $(**),$ on a en intégrant :
\begin{align*}
\vert P_{k+1}(x)+Q_{k+1}(x)-1\vert & \leq 2\int_{0}^{x}\frac{(2t)^{k}}{k!}dt\\
& \leq \frac{(2x)^{k+1}}{(k+1)!}.
\end{align*}

Ainsi, les séries de terme général respectifs $\displaystyle \|P_{k}-P_{k+1}\|_{\infty,[0,1]}$ et  $\displaystyle \|P_{k+1}+Q_{k+1}-1\|_{\infty,[0,1]}$  sont convergentes et donc la suite $(P_{k})$ converge uniformément vers $P$ sur $[0,1]$ et la suite  $(Q_{k})$ converge uniformément vers $Q$ sur $[0,1]$ (car $\displaystyle \left(\mathcal{C}^{0}([0,1],\mathbb{R}),\|.\|_{\infty,[0,1]}\right)$ est un espace de Banach). 

De plus, on a $$ P+Q=1 \mbox{ sur } [0,1].$$

En passant à la limite, il vient également pour tout $x\in[0,1]$ : $$\displaystyle P(x)=\left(1-x+\int_{0}^{x}Q(t)dt\right)^{2} \mbox{ et } Q(x)=1-\left(1-\int_{0}^{x}P(t)dt\right)^{2}.$$

\textbf{Remarque: } Plutôt que d'utiliser un argument de complétude, on peut dans le programme de prépa remarquer la convergence absolue des séries en jeu à $x$ fixé. Ainsi, il y a convergence simple des suites $(P_{k})$ et $(Q_{k})$ en tout point de $[0,1].$ Les deux dernières relations fonctionnelles s'obtiennent par convergence dominée car les suites de fonctions en jeu sont bornées uniformément, ce qui est une domination valable sur un segment.

La limite cherchée est alors (en utilisant $P+Q=1$ et en passant à la limite) $$ I=\int_{0}^{1}\left(1-x+\int_{0}^{x}Q(t)dt\right)^{3}=\int_{0}^{1}\left(1-\int_{0}^{x}P(t)dt\right)^{3}.$$

Il ne reste plus qu'à déterminer $\displaystyle \phi(x)=\int_{0}^{x}P(t)dt.$
Or, on sait $$\displaystyle P(x)=\left(1-\int_{0}^{x}P(t)dt\right)^{2} \mbox{ i.e. } \phi'(x)=\left(1-\phi(x)\right)^{2}.$$
En séparant les variables et en utilisant $\phi(0)=0,$ il vient $$ \frac{1}{1-\phi(x)}-1=x \mbox{ i.e. } \phi(x)=1-\frac{1}{1+x}=\frac{x}{1+x}$$ d'où $\displaystyle P(x)=\frac{1}{(1+x)^{2}}.$

Ainsi, on obtient finalement $$ I=\int_{0}^{1}\frac{dx}{(1+x)^{3}}=\left[\frac{-1}{2(1+x)^{2}}\right]_{0}^{1}=\frac{3}{8}.$$