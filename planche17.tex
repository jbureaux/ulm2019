\chapter{Planche 17}

\section{Sujet}

\paragraph{Exercice}

\section{Solution}

Soit $f$ une fonction continue sur $[0,1].$

L'identité à prouver se résume à montrer : $$I_{1}+I_{2}:=\int_{-\frac{1}{2}}^{0}xf(3x^{2}-2x^{3})dx+\int_{1}^{\frac{3}{2}}xf(3x^{2}-2x^{3})dx=\int_{0}^{1}xf(3x^{2}-2x^{3})dx:=J.$$

On utilise une méthode ad-hoc, en utilisant le changement de variables suivant : $3-2x=4\cos^{2}(\theta).$

D'une part, $$\mbox{Re}[(e^{i\theta})^{3}]=\mbox{Re}[\left(\cos(\theta)+i\sin(\theta)\right)^{3}]=\cos^{3}(\theta)-3\sin^{2}(\theta)\cos(\theta)=4\cos^{3}(\theta)-3\cos(\theta).$$

Ainsi, $$3x^{2}-2x^{3}=x^{2}(3-2x)=\cos^{2}(\theta)\left(3-4\cos^{2}(\theta)\right)^{2}=\cos^{2}(3\theta).$$

Et, $$xdx=-\frac{3-4\cos^{2}(\theta)}{2}\times 4\cos(\theta)\sin(\theta)d\theta=-2\sin(\theta)\cos(3\theta)d\theta.$$

Il vient donc par le changement de variables précedemment indiqué:
\begin{align*}
-\frac{I_{1}+I_{2}}{2} & = \int_{0}^{\frac{\pi}{6}}\sin(\theta)\cos(3\theta)f(\cos^{2}(3\theta))d\theta+\int_{\frac{\pi}{3}}^{\frac{\pi}{2}}\sin(\theta)\cos(3\theta)f(\cos^{2}(3\theta))d\theta\\
& =-\int_{\frac{\pi}{6}}^{\frac{\pi}{3}}\sin(\frac{\pi}{3}-\alpha)\cos(3\alpha)f(\cos^{2}(3\alpha))d\alpha+ \int_{\frac{\pi}{6}}^{\frac{\pi}{3}}\sin(\frac{2\pi}{3}-\alpha)\cos(3\alpha)f(\cos^{2}(3\alpha))d\alpha\\
& \mbox{ par le changement de variables dans la première intégrale : } \alpha=\frac{\pi}{3}-\theta\\
& \mbox{ puis par le changement de variables dans la deuxième intégrale : } \alpha=\frac{2\pi}{3}-\theta.\\
& =(\frac{1}{2}+\frac{1}{2})\int_{\frac{\pi}{6}}^{\frac{\pi}{3}}\sin(\alpha)\cos(3\alpha)f(\cos^{2}(3\alpha))d\alpha\\
& \mbox{ en développant la partie en } \sin  \mbox{ dans chacune des intégrales}\\
& = -\frac{J}{2}\\
& \mbox{ par le changement de variables indiqué. }
\end{align*}

On obtient donc bien l'identité désirée.