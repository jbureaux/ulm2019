\chapter{Mesure uniforme et tiré en arrière exotique}

\section{Sujet}

\paragraph{Exercice }

\section{Solution du premier exercice}

Soit $f$ une fonction continue sur $[0,1].$

L'identité à prouver se résume à montrer : $$I_{1}+I_{2}:=\int_{-\frac{1}{2}}^{0}xf(3x^{2}-2x^{3})dx+\int_{1}^{\frac{3}{2}}xf(3x^{2}-2x^{3})dx=\int_{0}^{1}xf(3x^{2}-2x^{3})dx:=J.$$

On utilise une méthode ad-hoc, en utilisant le changement de variables suivant : $3-2x=4\cos^{2}(\theta).$

D'une part, $$\mbox{Re}[(e^{i\theta})^{3}]=\mbox{Re}[\left(\cos(\theta)+i\sin(\theta)\right)^{3}]=\cos^{3}(\theta)-3\sin^{2}(\theta)\cos(\theta)=4\cos^{3}(\theta)-3\cos(\theta).$$

Ainsi, $$3x^{2}-2x^{3}=x^{2}(3-2x)=\cos^{2}(\theta)\left(3-4\cos^{2}(\theta)\right)^{2}=\cos^{2}(3\theta).$$

Et, $$xdx=-\frac{3-4\cos^{2}(\theta)}{2}\times 4\cos(\theta)\sin(\theta)d\theta=-2\sin(\theta)\cos(3\theta)d\theta.$$

Il vient donc par le changement de variables précedemment indiqué:
\begin{align*}
-\frac{I_{1}+I_{2}}{2} & = \int_{0}^{\frac{\pi}{6}}\sin(\theta)\cos(3\theta)f(\cos^{2}(3\theta))d\theta+\int_{\frac{\pi}{3}}^{\frac{\pi}{2}}\sin(\theta)\cos(3\theta)f(\cos^{2}(3\theta))d\theta\\
& =-\int_{\frac{\pi}{6}}^{\frac{\pi}{3}}\sin(\frac{\pi}{3}-\alpha)\cos(3\alpha)f(\cos^{2}(3\alpha))d\alpha+ \int_{\frac{\pi}{6}}^{\frac{\pi}{3}}\sin(\frac{2\pi}{3}-\alpha)\cos(3\alpha)f(\cos^{2}(3\alpha))d\alpha\\
& \mbox{ par le changement de variables dans la première intégrale : } \alpha=\frac{\pi}{3}-\theta\\
& \mbox{ puis par le changement de variables dans la deuxième intégrale : } \alpha=\frac{2\pi}{3}-\theta.\\
& =(\frac{1}{2}+\frac{1}{2})\int_{\frac{\pi}{6}}^{\frac{\pi}{3}}\sin(\alpha)\cos(3\alpha)f(\cos^{2}(3\alpha))d\alpha\\
& \mbox{ en développant la partie en } \sin  \mbox{ dans chacune des intégrales}\\
& = -\frac{J}{2}\\
& \mbox{ par le changement de variables indiqué. }
\end{align*}
On obtient donc bien l'identité désirée.

$\underline { \text{Une autre solution}}$. %LOU16

Soit $u$ la fonction $x \longmapsto 3x^2 -2x^3.$ L'examen des variations de $u$ montre que $u$ réalise une bijection de chacun des intervalles
$[ -\frac {1}{2};0],\:[0;1],\:[1;\frac {3}{2}]\:$ sur $\: [0;1]$.
Notons $v_1,v_2,v_3$ les bijections réciproques des .précédentes. Elles sont  $\mathcal C ^1$ sur $]0;1[$.
Alors: $\forall t \in [0;1],\:\: v_1(t), v_2(t), v_3(t)\:$ sont les trois solutions (réelles) de l'équation $u(x) = t,$ équivalente à $2x^3 -3x^2 +t =0.$ 
Les relations entre coefficients et racines entraînent alors: $\forall t \in [0:1],\:\: v_1(t)^2 +v_2(t)^2 +v_3(t)^2 = \left(\frac 32\right) ^2,\:$ puis

$\forall t \in ]0;1[,\:\:\:  v_1(t)v_1'(t)+ v_2(t)v_2'(t) + v_3(t)v_3'(t) = 0.$ 
En effectuant respectivement sur chacun des intervalles $[-\frac12;0],\:[0;1],\: [1;\frac 32]$  le changement de variable $x=v_1(t),\:x=v_2(t),\:x=v_3(t),\:$  on obtient:

$\displaystyle  2\int_0 ^1 x(f\circ  u)(x)\: \mathrm d x - \int_{-\frac 12}^{\frac 32} x(f\circ u) (x) \:\mathrm d x = \int_0^1 \left(v_1v_1'
+ v_2v_2'+v_3v_3' \right) (t) \:\mathrm d t =0.$ 

\section{ Solution du deuxième exercice} %LOU16

Notons $a_{ij}, \:b_{ij}$ les coefficients de $A$ et de $B$, et ${A_{ij}}, {B_{ij}}$ ceux des comatrices $\text {com}(A)$ et $\text{com}(B).$
En développant le déterminant de $A_j$ par rapport à la première colonne et celui de $B_j$ par rapport à la $j$-ième colonne, on obtient:

     $\displaystyle \sum _{j=1}^n \det {A_j} \det {B_j} = \sum _{j=1}^n\left(\sum _{i=1}^n  b_{i j}{A_{i 1}}\:\:\sum _{k=1}^n  a_{k1} {B_{k j}}\right)= \sum _{i,k =1}^n a_{k1}{A_{i1}}\:\sum_{j=1}^n b_{ij }{B_{kj}}$. 
 
 Or, en désignant par $\widehat{B_{ik}}$ la matrice obtenue en remplaçant dans $B$ la $k$-ième ligne de $B$ par la $i$-ième ligne de $B$:
 
 $\quad \displaystyle \sum _{j=1}^n  b_{ij}{B_{kj}}= \det {\widehat{B_{ik}}}= \left\{\begin{array} {cl} \det B & \text{si}\:\: i=k\\ 0& \text{sinon} \end{array} \right.$
 
 Ainsi: $\displaystyle \sum _{j=1}^n \det {A_j} \det {B_j}= \det{B}\sum _{k=1}^n a_{k1} A_{k1} = \det {A}\det{B}.$
