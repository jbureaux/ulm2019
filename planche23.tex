\chapter{Autour d'un problème d'Erd{\H{o}}s}

\section{Sujet}

\paragraph{Exercice}
Soit $(X_i)_{i\geqslant 0}$ des variables aléatoires indépendantes de loi uniforme sur $\{-1,1\}$. On fixe $\alpha \in \mathopen]0,1\mathclose[$.
Montrer que la fonction
\[
\begin{array}{cccc}
F \ :  &\mathbb R & \longrightarrow & [0,1]\\
& t & \longmapsto & \displaystyle\mathbb P\left(\sum_{k=0}^\infty \alpha^k X_k \leqslant t\right)
\end{array}
\]
est continue.


\section{Solution de l'exercice}

Soit $\alpha\in]0,1[.$

Notons $\displaystyle Z_{\alpha}=\sum_{k\geq 0}\alpha^{k}X_{k}$ où les $(X_{k})$ sont une suite de Rademacher i.i.d.

Ensuite, on note $\displaystyle F_{\alpha}$ la fonction de répartition de $Z_{\alpha}.$

On note également $\mu_{\alpha}$ la loi de $Z_{\alpha}$ et on introduit $$K_{\alpha}=\left\{ \sum_{k\geq 0}\alpha^{k}\varepsilon_{k}\mbox{ }|\mbox{ } \forall k\geq 0,\mbox{ } \varepsilon_{k}\in\{-1,1\}  \right\}$$ le support de $\mu_{\alpha}$, qui est un compact de $\displaystyle [-\frac{1}{1-\alpha},\frac{1}{1-\alpha}]$ (en procédant par extraction diagonale).

\begin{enumerate}
\item \underline{Montrons que tout élément de $K_{\alpha}$ s'écrit de manière unique si $0<\alpha<\frac{1}{2}.$}
    
(on peut montrer en calculant la transformée de Fourier de $\mu_{\frac{1}{2}}$ -i.e. la fonction caractéristique de $Z_{\alpha}$- que $\mu_{\frac{1}{2}}$ est la mesure de Lebesgue sur $[-2,2]$ et donc $K_{\frac{1}{2}}=[-2;2]$).

Soit $t\in K_{\alpha}.$ 

Supposons qu'il existe deux suites $\varepsilon,\tilde{\varepsilon}\in\{-1,1\}^\mathbb{N}$ telles que $\varepsilon\neq \tilde{\varepsilon}$ vérifiant $\displaystyle \sum_{k\geq 0}\alpha^{k}\varepsilon_{k}=t=\sum_{k\geq 0}\alpha^{k}\tilde{\varepsilon_{k}}.$

Notons $k_{0}=\min\left\{ k\geq 0\mbox{ }|\mbox{ } \varepsilon_{k}\neq \tilde{\varepsilon_{k}} \right\}.$ On a alors $\displaystyle \alpha^{k_{0}}\vert \varepsilon_{k_{0}}-\tilde{\varepsilon_{k_{0}}}\vert =\big\vert \sum_{k\geq k_{0}+1}\alpha^{k}\left(\varepsilon_{k}-\tilde{\varepsilon_{k}}\right)\big\vert.$

Par l'inégalité triangulaire, il vient $\displaystyle 2\alpha^{k_{0}}\leq 2\sum_{k\geq k_{0}+1}\alpha^{k} \mbox{ i.e. } 2\alpha^{k_{0}}\leq \frac{2\alpha^{k_{0}+1}}{1-\alpha}.$
On a ainsi $\displaystyle 1\leq \frac{\alpha}{1-\alpha}$ ce qui donne $\alpha\geq \frac{1}{2}.$ Ceci est impossible compte-tenu du choix de $\alpha.$\\

\textbf{Remarque :} Ceci démontre que $F_{\alpha}$ est continue si $\alpha<\frac{1}{2}.$\\

En effet, soit $0<\alpha<1/2.$ 

Soit $t=\sum\limits_{k\geq 0}\varepsilon_{k}\alpha^{k}\in K_{\alpha}.$

Cette écriture étant unique, on a : $$\mathbb{P}(Z_{\alpha}=t)=\mathbb{P}\left( \bigcap_{k\geq 0}\{X_{k}=\varepsilon_{k}\} \right)=0.$$

Ainsi, pour tout $t\in\mathbb{R}$ (distinguer les cas $t\in K_{\alpha}$ et $t\notin K_{\alpha}$) : $$\lim_{x\rightarrow t^{+}}F_{\alpha}(x)-\lim_{x\rightarrow t^{-}}F_{\alpha}(x)=P(Z_{\alpha}=t)=0$$ et $F_{\alpha}$ est ainsi continue.\\ 


%***Pour une preuve plus directe, on peut aussi procéder ainsi.

%\underline{Continuité de $F_{\alpha}$ lorsque $\alpha\in]0,\alpha_{0}[$} 

%où $0<\alpha_{0}<\frac{1}{2}$ est à déterminer!

%\textbf{Remarque :} Le point clé est d'obtenir la continuité des $F_{\alpha}$ lorsque $\alpha\in]0,l[$ pour un certain $0<l<1.$

%Soit $t\in\mathbb{R}.$
%\begin{itemize}
%\item \underline{ou $t\notin %K_{\alpha}$: } 

%vu que $K_{\alpha}$ est compact alors $\displaystyle d(t,K_{\alpha})>0$ et alors il existe $0<\varepsilon\ll1$ tel que $\displaystyle [t-\varepsilon,t+\varepsilon]\cap K_{\alpha}=\emptyset.$

%Ainsi, pour tout $\vert h\vert \leq \varepsilon,$ on a $\displaystyle F_{\alpha}(t+h)=F_{\alpha}(t)$ et donc $F_{\alpha}$ est continue en $t$ (car localement constante dans un voisinage de $t$).

%\item \underline{ou $t\in K_{\alpha}$ :} 

%Soit $l\gg1.$ On écrit alors $\displaystyle t=\sum_{k\geq 0}\alpha^{k}\varepsilon_{k}=\sum_{k=0}^{l}\alpha^{k}\varepsilon_{k}+\sum_{k\geq l+1}\alpha^{k}\varepsilon_{k}:=\tilde{t}+\sum_{k\geq l+1}\alpha^{k}\varepsilon_{k}$ où pour tout $k\geq 0,$ $\varepsilon_{k}\in\{-1,1\}.$

%Soit $\displaystyle \vert h \vert \leq \frac{\alpha^{l+1}}{1-\alpha}.$ On a alors : 
%\begin{align*}
%\vert F_{\alpha}(t+h)-F_{\alpha}(t)\vert & \leq \mathbb{P}\left( \vert X_{\alpha} -t\vert \leq \frac{\alpha^{l+1}}{1-\alpha} \right)\\
%& \leq \mathbb{P}\left( \vert X_{\alpha}-\tilde{t}\vert \leq \frac{2\alpha^{l+1}}{1-\alpha} \right)\\
%& \mbox{ par l'inégalité triangulaire, en majorant le reste } \vert t-\tilde{t}\vert\\
%& := \mathbb{P}(A_{l}).
%\end{align*}

%\item Montrons que $\displaystyle A_{l}\subset \bigcap_{k=0}^{l}\{ X_{k}=\varepsilon_{k} \}$. 

%Supposons que ce ne soit pas le cas. 

%Il existe alors une réalisation $\omega\in A_{l}$ et un indice $k\in\{0,\ldots,l\}$ tel que $X_{k}(\omega)\neq \varepsilon_{k}(\omega).$ 

%On note alors $k_{0}$ l'indice minimal vis à vis de cette dernière propriété.

%On obtient alors par l'inégalité triangulaire inversée, la chaîne suivante d'inégalités :
%$$2\alpha^{k_{0}}-\frac{2\alpha^{k_{0}+1}}{1-\alpha}\leq \vert X_{\alpha}(\omega)-\tilde{t}\vert \leq \frac{2\alpha^{l+1}}{1-\alpha}.$$
%Il vient alors en simplifiant $\displaystyle \alpha^{l+1-k_{0}}\geq 1-2\alpha.$

%Comme $k_{0}\in\{0,\ldots,l\},$ on a la minoration suivante : $\displaystyle \alpha^{l+1-k_{0}}\leq \alpha$ (vu que $\alpha\in]0,1[).$

%Mais alors, on obtient $\displaystyle 3\alpha-1\geq 0.$ Ceci est impossible, si on pose $\alpha_{0}:=\frac{1}{3}$ et si on choisit $\alpha\in]0,\alpha_{0}[.$

%On obtient  ainsi
%\begin{align*}
%\vert F_{\alpha}(t+h)-F_{\alpha}(t)\vert & \leq   \mathbb{P}\left( \bigcap_{k=0}^{l-1}\{ X_{k}=\varepsilon_{k} \} \right)\\
%& \leq \frac{1}{2^{l}} \mbox{ par indépendance des va en jeu.}
%\end{align*}

%Ainsi, $F_{\alpha}$ est continue en $t$ (il est alors aisé à ce stade de vérifier la définition de la continuité de $F_{\alpha}$ au point $t$).

%Ainsi, $F_{\alpha}$ est continue sur $\mathbb{R}$ et donc $\mu_{\alpha}$ est sans atomes.

%\textbf{Remarque  :} Enfin, on peut optimiser la preuve précédente pour la faire fonctionner pour tout $\alpha<\frac{1}{2}$ (et pas seulement dans un voisinage de $0$).

%Avec les notations précédentes, il suffit de prouver l'inclusion $\displaystyle A_{l}\subset \bigcap_{k=0}^{l+1-B}\{X_{k}=\varepsilon_{k}\}$ où $B$ est un entier suffisamment grand de manière à avoir $\displaystyle\alpha^{B}\leq \varepsilon$ alors que $\displaystyle \alpha\leq \frac{1-\varepsilon}{2}.$
%\end{itemize}

\item \underline{Astuce de convolution}

On a écrit alors : $$Z_{\alpha}=\sum_{k\geq 0}(\alpha^{2})^{k}X_{2k}+\alpha\sum_{k\geq 0}(\alpha^{2})^{k}X_{2k+1}=Z_{1,\alpha^{2}}+\alpha Z_{2,\alpha^{2}}$$ où $Z_{1,\alpha^{2}}$ et $Z_{2,\alpha^{2}}$ sont des copies indépendantes et de même loi que $Z_{\alpha^{2}}.$

Ainsi, $\mu_{\alpha}$ se présente comme la convolution de deux mesures qui sont sans atomes si $\alpha^{2}<\frac{1}{2}$ (par le premier point de la preuve). Ainsi, $\mu_{\alpha}$ est également sans atomes si $\alpha^{2}<\frac{1}{2}$ et donc $F_{\alpha}$ est continue sur $\mathbb{R}$ si $\alpha^{2}<\frac{1}{2}.$

Par une récurrence directe, on obtient alors que pour tout $\alpha\in]0,1[,$ $F_{\alpha}$ est continue sur $\mathbb{R}.$

\end{enumerate}


***\textbf{Remarque :} C'est un problème ouvert et difficile, connu comme étant une conjecture d'Erdös de savoir si la loi de $Z_{\alpha}$ (pour $\alpha\geq \frac{1}{2}$) est absolument continue par rapport à Lebesgue sauf éventuellement lorsque $\alpha$ est l'inverse d'un nombre de Pisot.\\

On peut se référer au papier suivant pour voir l'état de l'art :

http://u.math.biu.ac.il/~solomyb/RESEARCH/sixty.pdf\\

Enfin, il est à noter qu'un résultat récent montre que le support de $Z_{\alpha}$ est de dimension de Hausdorff $1$ pour tout nombre transcendant $\alpha\in\,]\frac{1}{2},1[.$ Voici la référence :

http://export.arxiv.org/pdf/1810.08905
