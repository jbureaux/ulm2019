\chapter[Fonction presque multiplicative]{Fonction presque multiplicative sur un groupe}

\section{Sujet}

\paragraph{Exercice}
Soit $G$ un groupe, $\delta > 0$ et $f : G \to \mathbb C$ une fonction telle que
\[
\forall x,y \in G,\quad |f(xy)-f(x)f(y)| \leqslant \delta.
\]
Montrer qu'il existe $C > 0$ tel que $|f(x)| \leqslant C$ pour tout $x \in G$ ou $f(xy) = f(x)f(y)$ pour tous $x,y \in G$.
Trouver la plus petite valeur de $C$ possible.

\paragraph{Deuxième exercice}

Soit $f_n : \mathbb R \to \mathbb R$ une suite de fonctions dérivables telles que $\|f_n'\|_\infty \leqslant 1$ pour tout $n\geqslant 1$. On suppose que $f_n$ converge simplement vers $g$. Montrer que $g$ est continue.

\paragraph{Exercice additionnel}
Peut-on trouver un sous-groupe strict de $(\mathbb Q,+)$ non monogène ?

\section{Solution de l'exercice} % BobbyJoe

Soit $f$ un quasi-morphisme multiplicatif (quasi-caractère?) de $G$ dans $\mathbb{C}.$\\

\begin{itemize}
\item Ou il existe une constante $C>0$ telle que pour tout $x\in \mathbb{G},$ $\displaystyle \vert f(x) \vert \leq C.$
\item Ou il existe une suite d'éléments de $G$ : $\displaystyle (x_{n})\in G^{\mathbb{N}}$ telle que $\displaystyle \lim_{n\rightarrow +\infty} \vert f(x_{n})\vert =+\infty.$
\end{itemize}

Soit $x,y\in G.$ 

On note alors pour $n\gg 1,$ $a_{n}=\vert f(x_{n})\vert.$
On a alors d'une part pour $n\gg1,$ $$B_{n}:=\left \vert \frac{f(xyx_{n})}{f(x_{n})}-f(xy) \right\vert =\frac{ \vert f(xyx_{n})-f(xy)f(x_{n}) \vert }{a_{n}}\leq \frac{\delta}{a_{n}}\longrightarrow_{n\rightarrow +\infty} 0.$$

Et d'autre part, on a également pour $n\gg 1,$ 
\begin{align*}
C_{n}:=\left\vert \frac{f(xyx_{n})}{f(x_{n})}-f(x)f(y) \right\vert & \leq \frac{ \vert f(xyx_{n})-f(x)f(yx_{n})\vert }{a_{n}}+ \left\vert \frac{f(x)f(yx_{n})}{f(x_{n})}-f(x)f(y) \right\vert\\
& \leq \frac{\delta}{a_{n}}+ \frac{\vert f(x) \vert }{a_{n}}\vert f(yx_{n})-f(y)f(x_{n})\vert\\
& \leq \frac{\delta(1+\vert f(x) \vert )}{a_{n}}\longrightarrow_{n\rightarrow +\infty} 0.
\end{align*}

Ainsi, par l'inégalité triangulaire, on a $$\vert f(xy)-f(x)f(y)\vert \leq B_{n}+C_{n}\longrightarrow_{n\rightarrow +\infty}0.$$

D'où le fait que $f$ est un morphisme multiplicatif (caractère) de $G$ i.e. $\displaystyle f(xy)=f(x)f(y).$

\paragraph{Autre solution.} % Siméon

Soient $(x,y,z) \in G^2$. On dispose par hypothèse de $\theta_1,\theta_2,\theta_3$ complexes de modules majorés par $\delta$ tels que \(f(xyz) = f(xy)f(z) + \theta_1\) et :
\[
f(xyz) = f(x)f(yz) + \theta_2 = f(x)\bigl(f(y)f(z) + \theta_3 \bigr) + \theta_2 ,
\]
donc l'inégalité triangulaire conduit facilement à :
\[
\Bigl|\bigl(f(xy)-f(x)f(y)\bigr)f(z) \Bigr| \leqslant \bigl(|f(x)| + 2\bigr) \delta.
\]
S'il existe $(x,y) \in G^2$ tels que $f(xy)\neq f(x)f(y)$, on obtient alors :
\[
\sup_{z\in G} |f(z)| \leqslant \frac{\bigl(|f(x)|+2\bigr)\delta}{|f(xy)-f(x)f(y)|}.
\]
Remarquons que ceci reste vrai pour toute loi de composition interne associative.



\section{Solution du deuxième exercice} % Siméon

Soit $x,y$ deux réels. D'après l'inégalité des accroissements finis :
\[
\forall n \geqslant 1,\quad |f_n(x) - f_n(y)| \leqslant \|f_n'\|_\infty\,|x-y| \leqslant |x-y|,
\]
d'où $|g(x) - g(y)| \leqslant |x-y|$ par passage à la limite. Ceci montre que la fonction $g$ est $1$-lipschitzienne, donc en particulier continue sur $\mathbb R$.

\paragraph{Remarque} Sous ces hypothèses, on peut montrer que la convergence est en fait uniforme sur tout segment $[a,b]$ de $\mathbb R$. Soit $\varepsilon > 0$, et $(x_0,x_1,\dots,x_k)$ une subdivision du segment $[a,b]$, de pas majoré par $\varepsilon/2$. On dispose d'un rang $N$ tel que :
\[
\forall n\geqslant N,\forall i \in \{0,\dots,k\},\quad |f_n(x_i) - g(x_i)| < \frac{\varepsilon}2.
\]
Pour tout élément $x \in [a,b]$, on peut trouver $i \in \{1,\dots,k\}$ tel que $|x-x_i|\leqslant \varepsilon/4$. Puisque $g$ et les $f_n$ sont $1$-lipschitziennes, l'inégalité triangulaire entraîne alors :
\[
\forall n \geqslant N,\forall x \in [a,b],\quad |f_n(x) - g(x)| < \frac{\varepsilon}2 + \frac{\varepsilon}4 + \frac{\varepsilon}4,
\]
d'où le résultat annoncé.
Ces conclusions s'apparentent à un cas simple du théorème d'Ascoli (hors-programme), qui caractérise les parties compactes de l'ensemble des fonctions continues sur $[a,b]$ pour la norme $\|\cdot\|_\infty$.

\section{Solution de l'exercice additionnel} % Siméon

L'ensemble $\mathbb D$ des nombres décimaux est un sous-groupe de $(\mathbb Q,+)$. Il est dense dans $\mathbb R$ donc non monogène. 