\chapter{Autour de la densité logarithmique}

\section{Sujet}

\paragraph{Exercice}
Soit $(a_n)_{n\geqslant1}$ une suite de nombres entiers strictement positifs, strictement croissante. On pose $A(x) = \mathrm{Card}(\{k \geqslant 1 : a_{k} \leqslant  x\})$. A-t-on \[\sum _{k\geqslant1} \frac{1}{a_{k}} <\infty  \iff  \frac{A(x)}{x} \xrightarrow[x\rightarrow \infty ]{} 0 ?\]

\paragraph{Deuxième exercice}

Soit $n\geqslant 1$ un entier, $A \in \mathscr M_n(\mathbb C)$ une matrice sans valeurs propres multiples et $p \geqslant 2$ un entier. Résoudre l'équation suivante, d'inconnue $X \in \mathscr M_n(\mathbb C)$ :
\[
    XA - AX = X^p.
\]

\section{Solution du premier exercice}

L'équivalence demandée est \textbf{fausse}. Voici deux contre-exemples :
\begin{enumerate}
    \item On peut penser à utiliser $a_{k}=p_{k}$ où $\displaystyle (p_{k})_{k\geq 1}$ désigne la suite des nombres premiers. Avec les notations de l'énoncé, on a par le théorème des nombres premiers $$A(x)\sim \frac{x}{\ln(x)} \mbox{ et alors } \frac{A(x)}{x}\longrightarrow_{x\rightarrow +\infty} 0.$$
    Cependant, on sait que la série de terme général $\displaystyle \frac{1}{p_{k}}$ est divergente.
    \item Autre possibilité : $a_k = \lfloor k \ln(k)\rfloor$ si $k>1$ et $a_1=1$. En effet, si l'on note pour tout $x>0$, $k_x$ l'entier tel que $a_{k_x}\leqslant x <a_{k_x+1}$, on a \[\frac{A(x)}{x} \leqslant \frac{A(a_{k_x+1})}{a_{k_x}} \leqslant \frac{k_x+1}{k_x\ln(k_x)} \xrightarrow[x\rightarrow \infty ]{} 0\]
    tandis que $\displaystyle \sum_{k>1} \frac1{a_k} =+\infty$.
\end{enumerate}

En revanche, une des implications est toujours vraie. Prouvons la tout de même.

On procède alors à une transformation d'Abel (en utilisant le formalisme des intégrales de Lebesgue-Stieljes).

On a pour $x\geq 1,$
$$\sum_{1\leq k\leq x}\frac{1}{a_{k}} = \int_{1}^{x}\frac{dA(t)}{t}.$$

Ainsi, on obtient pour $y>x\geq 1$ et par monotonie des intégrandes :
\begin{align*} 
\sum_{x<k\leq y}\frac{1}{a_{k}} & =\int_{x}^{y}\frac{dA(t)}{t}.\\
& \geq \frac{1}{y}\int_{x}^{y}dA(t)\\
& \geq \frac{A(y)-A(x)}{y}.\\
\mbox{ Et ainsi, } \frac{A(y)}{y} & \leq \sum_{x<k\leq y}\frac{1}{a_{k}}+ \frac{A(x)}{y}.
\end{align*}

Il vient alors $$\limsup_{y\rightarrow +\infty}\frac{A(y)}{y} \leq \sum_{k>x}\frac{1}{a_{k}}.$$

Et en faisant tendre $x$ vers $+\infty$, il vient le résultat escompté (par encadrement, la quantité en jeu est positive) : $$\lim_{y\rightarrow +\infty}\frac{A(y)}{y}=0.$$

\paragraph{Remarque} % Siméon
On dispose cependant toujours de l'égalité suivante dans $[0,+\infty]$ :
\[
\sum_{k\geqslant 1} \frac1{a_k} = \int_0^{+\infty} \frac{A(x)}{x^2}\,dx.
\]
En particulier, la somme converge si et seulement si l'intégrale converge.

\section{Solution du deuxième exercice}

On sait que le spectre de $A$ est constitué de $n$ éléments distincts deux à deux donc $A$ est diagonalisable sur $\mathbb{C}.$

On écrit donc $A=PDP^{-1}$ où $D$ est une matrice diagonale dont les éléments (sur la diagonale) sont deux à deux distincts.

On peut alors se ramener à résoudre  l'équation matricielle suivante (quitte à changer l'inconnue $X$ en $P^{-1}XP$) : $$DX-XD=X^{p}.$$

Une récurrence directe montre alors que pour tout $k\in \mathbb{N},$ $$DX^{k}-X^{k}D=kX^{p+k-1}.$$

Ainsi, en passant à la trace les différentes relations précédentes, on obtient par une caractérisation bien connue des matrices nilpotentes : $X$ est nilpotente.

Supposons $X$ non nulle et notons $l\geq 2$ l'indice de nilpotence de $X.$

En regardant la relation pour $k=l-1,$ il vient (vu que $p\geq 2$) : $$DX^{l-1}=X^{l-1}D.$$

Enfin, par un calcul matriciel direct, on obtient que le commutant de $D$ est formé des matrices diagonales (car les éléments de la diagonale de $D$ sont deux à deux distincts).

Comme $X^{l-1}$ commute avec $D$ alors $X^{l-1}$ est diagonale. Cependant, $X$ est nilpotente alors $X^{l-1}$ l'est aussi. Ainsi, $X^{l-1}=0$ et ceci contredit la minimalité de $l.$

Ainsi, l'unique solution de l'équation matricielle proposée est la matrice nulle.