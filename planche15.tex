\chapter{Planche 15}

\section{Sujet}

\section{Solution de l'exercice}

L'équivalence demandée est \textbf{fausse}. 

On peut penser à utiliser $a_{k}=p_{k}$ où $\displaystyle (p_{k})_{k\geq 1}$ désigne la suite des nombres premiers.

Avec les notations de l'énoncé, on a par le théorème des nombres premiers $$A(x)\sim \frac{x}{\ln(x)} \mbox{ et alors } \frac{A(x)}{x}\longrightarrow_{x\rightarrow +\infty} 0.$$

Cependant, on sait que la série de terme général $\displaystyle \frac{1}{p_{k}}$ est divergente.\\

En revanche, une des implications est toujours vraie. Prouvons la tout de même.

On procède alors à une transformation d'Abel (en utilisant le formalisme des intégrales de Lebesgue-Stieljes).

On a pour $x\geq 1,$
$$\sum_{k\geq x}\frac{1}{a_{k}} = \int_{1}^{+\infty}\frac{dA(t)}{t}.$$

Ainsi, on obtient pour $y>x\geq 1$ et par monotonie des intégrandes :
\begin{align*} 
\sum_{x\leq k\leq y}\frac{1}{a_{k}} & =\int_{x}^{y}\frac{dA(t)}{t}.\\
& \geq \frac{1}{y}\int_{x}^{y}dA(t)\\
& \geq \frac{A(y)-A(x)}{y}.\\
\mbox{ Et ainsi, } \frac{A(y)}{y} & \leq \sum_{x\leq k\leq y}\frac{1}{a_{k}}+ \frac{A(x)}{y}.
\end{align*}

Il vient alors $$\limsup_{y\rightarrow +\infty}\frac{A(y)}{y} \leq \sum_{k\geq x}\frac{1}{a_{k}}.$$

Et en faisant tendre $x$ vers $+\infty$, il vient le résultat escompté (par encadrement, la quantité en jeu est positive) : $$\lim_{y\rightarrow +\infty}\frac{A(y)}{y}=0.$$