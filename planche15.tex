\chapter{Planche 15}

\section{Sujet}

\paragraph{Premier exercice}
Soit $(a_n)_{n>1}$ une suite de nombres entiers strictement positifs, strictement croissante. On pose $A(x) = \mathrm{Card}(\{k > 1 : a_{k} \leqslant  x\})$. A-t-on \[\sum _{k>1} \frac{1}{a_{k}} <\infty  \Longleftrightarrow  \frac{A(x)}{x} \xrightarrow[x\rightarrow \infty ]{} 0 ?\]

\section{Solution de l'exercice}

L'équivalence demandée est \textbf{fausse}. Voici deux contre-exemples :
\begin{enumerate}
    \item On peut penser à utiliser $a_{k}=p_{k}$ où $\displaystyle (p_{k})_{k\geq 1}$ désigne la suite des nombres premiers. Avec les notations de l'énoncé, on a par le théorème des nombres premiers $$A(x)\sim \frac{x}{\ln(x)} \mbox{ et alors } \frac{A(x)}{x}\longrightarrow_{x\rightarrow +\infty} 0.$$
    Cependant, on sait que la série de terme général $\displaystyle \frac{1}{p_{k}}$ est divergente.
    \item Autre possibilité : $a_k = \lfloor k \ln(k)\rfloor$ si $k>1$ et $a_1=1$. En effet, si l'on note pour tout $x>0$, $k_x$ l'entier tel que $a_{k_x}\leqslant x <a_{k_x+1}$, on a \[\frac{A(x)}{x} \leqslant \frac{A(a_{k_x+1})}{a_{k_x}} \leqslant \frac{k_x+1}{k_x\ln(k_x)} \xrightarrow[x\rightarrow \infty ]{} 0\]
    tandis que $\displaystyle \sum_{k>1} \frac1{a_k} =+\infty$.
\end{enumerate}

En revanche, une des implications est toujours vraie. Prouvons la tout de même.

On procède alors à une transformation d'Abel (en utilisant le formalisme des intégrales de Lebesgue-Stieljes).

On a pour $x\geq 1,$
$$\sum_{1\leq k\leq x}\frac{1}{a_{k}} = \int_{1}^{x}\frac{dA(t)}{t}.$$

Ainsi, on obtient pour $y>x\geq 1$ et par monotonie des intégrandes :
\begin{align*} 
\sum_{x<k\leq y}\frac{1}{a_{k}} & =\int_{x}^{y}\frac{dA(t)}{t}.\\
& \geq \frac{1}{y}\int_{x}^{y}dA(t)\\
& \geq \frac{A(y)-A(x)}{y}.\\
\mbox{ Et ainsi, } \frac{A(y)}{y} & \leq \sum_{x<k\leq y}\frac{1}{a_{k}}+ \frac{A(x)}{y}.
\end{align*}

Il vient alors $$\limsup_{y\rightarrow +\infty}\frac{A(y)}{y} \leq \sum_{k>x}\frac{1}{a_{k}}.$$

Et en faisant tendre $x$ vers $+\infty$, il vient le résultat escompté (par encadrement, la quantité en jeu est positive) : $$\lim_{y\rightarrow +\infty}\frac{A(y)}{y}=0.$$