\chapter{Planche 20}

\section{Sujet}

\paragraph{Exercice}

\section{Solution de l'exercice}

\begin{enumerate}
\item Introduisons la suite de Fibonacci ("shiftée") :
$$F_{1}=1,\mbox { }F_{2}=1 \mbox{ et } \forall n\geq 1,\mbox{ } F_{n+2}=F_{n}+F_{n+1}.$$

Le calcul des premiers termes de la suite $u$ nous invite à conjecturer : $$\forall n\geq 0,\mbox{ } u_{n+2}=\frac{s^{F_{n+1}}}{b_{n}},$$ où $\displaystyle b_{n}=\prod_{k=2}^{n}k^{F_{n+1-k}}.$

Prouvons-le par récurrence double.

\begin{itemize}
 \item \textbf{Initialisation :}

On a $u_{2}=s$ et $F_{1}=1;\mbox{ }b_{0}=1.$

Et, $u_{3}=s$ et $F_{2}=1;\mbox{ }b_{1}=1.$

\item \textbf{Hérédité : }

Supposons pour un certain $n\geq 0,$ $$u_{n+2}=\frac{s^{F_{n+1}}}{b_{n}} \mbox{ et } u_{n+3}=\frac{s^{F_{n+2}}}{b_{n+1}}.$$

Il vient alors 
\begin{align*}
u_{n+4} & =\frac{u_{n+2}u_{n+3}}{n+2}\\
& =\frac{s^{F_{n+1}+F_{n+2}}}{(n+2)b_{n}b_{n+1}}\\
& =s^{F_{n+3}}\times \frac{1}{(n+2)\prod_{k=2}^{n}k^{F_{n+1-k}}\prod_{k=2}^{n+1}k^{F_{n+2-k}}}\\
& = s^{F_{n+3}}\times \frac{1}{(n+1) \times (n+2)\times \prod_{k=2}^{n}k^{F_{n+1-k}+F_{n+2-k}}}\\
& =\frac{s^{F_{n+3}}}{(n+1)\times (n+2)\times \prod_{k=2}^{n}k^{F_{n+3-k}}}\\
& =\frac{s^{F_{n+3}}}{\prod_{k=2}^{n+2}k^{F_{n+3-k}}}\\
& = \frac{s^{F_{n+3}}}{b_{n+2}}.
\end{align*}
\end{itemize}

\item Notons alors $$\phi=\frac{1+\sqrt{5}}{2} \mbox{ et } \tau=-\frac{1}{\phi}.$$
Puis, considérons $$l=\sum_{k=1}^{+\infty}\frac{\ln(k)}{\phi^{k}} \mbox{ et } C=e^{l}.$$

\textbf{But : } Montrons que $\displaystyle b_{n}\sim C^{F_{n+1}}.$

On rappelle tout d'abord : $$\forall n\geq 1,\mbox{ } F_{n}=\frac{\phi^{n}-\tau^{n}}{\sqrt{5}}.$$

Ainsi, il vient pour $n\gg 1,$ : 
\begin{align*}
\ln(b_{n}) & = \sum_{k=2}^{n}F^{n+1-k}\ln(k)\\
& =\frac{1}{\sqrt{5}}\sum_{k=1}^{n}\left(\phi^{n+1-k}-\tau^{n+1-k}\right)\ln(k)\\
& =\frac{\phi^{n+1}}{\sqrt{5}}\sum_{k=1}^{n}\frac{\ln(k)}{\phi^{k}}-\frac{1}{\sqrt{5}}\sum_{k=1}^{n}\tau^{n+1-k}\ln(k).
\end{align*}

Or, vu que $\tau\in]-1,0[,$ on a : $$\vert \sum_{k=1}^{n}\tau^{n+1-k}\ln(k) \vert \leq \sum_{k=1}^{n}\ln(k) \leq n\ln(n).$$

Et ainsi, on obtient  : $$\frac{\sqrt{5}\ln(b_{n})}{\phi^{n+1}}\longrightarrow_{n\rightarrow +\infty} l.$$

On tire alors de cette relation : $$\ln(b_{n})\sim l F_{n+1}.$$

\item Ainsi, si $s \neq C,$ il vient : 
$$\ln(u_{n+2})=\ln(s)F_{n+1}-\ln(b_{n})\sim F_{n+1}\left(\ln(s)-\ln(C)\right)=F_{n+1}\ln(\frac{s}{C}).$$

On conclut alors $\displaystyle \lim_{n\rightarrow +\infty} u_{n}= \left\{ \begin{array}{l}
+\infty \mbox{ si } s>C\\
0 \mbox{ si } 0<s<C.
\end{array}\right.$

Pour le cas limite : $s=C,$ on écrit par définition de $l$
\begin{align*}
\ln(u_{n}) & =lF_{n+1}-\ln(b_{n})\\
& = \frac{\phi^{n+1}}{\sqrt{5}}\sum_{k=n+1}^{+\infty}\frac{\ln(k)}{\phi^{k}}-l\frac{l}{\sqrt{5}}\tau^{n+1}+\frac{\tau^{n+1}}{\sqrt{5}}\sum_{k=1}^{n}\frac{\ln(k)}{\tau^{k}}.\\
& \sim \frac{\phi^{n+1}}{\sqrt{5}}\sum_{k=n+1}^{+\infty}\frac{\ln(k)}{\phi^{k}}\\
& \geq \frac{\phi}{\sqrt{5}(\phi-1)}\ln(n+1).
\end{align*}

Et ainsi, on obtient par comparaison pour $s=C,$ $\displaystyle \lim_{n\rightarrow +\infty}u_{n}=+\infty.$

\end{enumerate}

\section{Solution du deuxième exercice}

\begin{enumerate}
\item Supposons pour tout $k\in\{1,\ldots,n+1\},$ $$A^{k}X=B^{K}Y.$$

On a alors en procédant à des combinaisons linéaires des relations précédentes et en utilisant le théorème de Cayley-Hamilton :  $$O_{n}=A\chi_{A}(A)X=B\chi_{A}(B)Y.$$

Comme $B$ est inversible, il vient en simplifiant $$\chi_{A}(B)Y=O_{n}.$$

Cependant, par hypothèse, on a $$\chi_{A}(B)Y-\mbox{det}(A)Y=\chi_{A}(A)X-\mbox{det}(A)X=-\mbox{det}(A)X,$$ d'où l'on tire $$\mbox{det}(A)(X-Y)=O_{n}.$$

Comme $A$ est inversible (i.e. $\mbox{det}(A)\neq 0$), il vient effectivement $\displaystyle X=Y.$

\item Montrons que le choix de $n+1$ relations consécutives est optimal.\\

En effet, pour le choix $$A= \left( \begin{array}{ll}
1 & 0\\
0 & 2
\end{array} \right); B= \left( \begin{array}{ll}
\frac{1}{2} & \frac{1}{4}\\
1 & \frac{3}{2}
\end{array}\right); X= \left( \begin{array}{ll}
1 & 0\\
1 & 0
\end{array}\right) \mbox{ et }  Y= \left( \begin{array}{ll}
2 & 0\\
0 & 0
\end{array}\right),$$ on a pour tout $k\in\{1,\ldots,n\},$ $\displaystyle A^{k}X=B^{k}Y$ où $A$ et $B$ sont inversibles et pourtant, $X\neq Y.$


\end{enumerate}