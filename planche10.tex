\chapter{Planche 10}

\section{Sujet}
\paragraph{Exercice}
Déterminer le nombre de fois qu'il faut, en moyenne, lancer une pièce
successivement et indépendemment au hasard pour observer une suite d'un
nombre impair de « piles » suivi d'un « face ».

\paragraph{Deuxième exercice}

\section{Solution de l'exercice} % Siméon

Soit $p \in \left]0;1\right[$ la probabilité d'obtenir « pile » lors d'un lancer.
Considérons la suite de variables aléatoires $(R_k)_{k \geq 1}$ définie par
les rangs successifs des lancers à l'issue desquels la séquence de « piles » en
cours a une longueur impaire. En notant $T$ le plus petit entier $k$ tel que le lancer de rang $1 + R_k$ donne « face », le nombre moyen recherché est $E(1 + R_T)$.

Il est clair que $T$ suit la loi $\mathcal G(1-p)$, tandis que $G_1 = R_1$ suit la loi $\mathcal G(p)$.
De plus, pour tout $k \in \mathbb N^*$, la variable aléatoire $G_{k+1} = R_{k+1} - R_k - 1$ suit la même loi que $G_1$.
Par télescopage, on obtient alors :
\[
1 + R_T = \sum_{k=1}^{T} (1 + G_k).
\]

En remarquant que $T$ est indépendante de chacune des variables $(G_k)$, on conclut à l'aide du lemme classique suivant dont la démonstration est rappelée brièvement ci-dessous.
\[
E(1 + R_T) =
E(T)E(1 + G_1) = \frac1{1-p}\left(1+\frac1p\right).
\]

\begin{lemme}[Formule de Wald]
Soient $(X_k)_{k\geqslant 1}$ une suite de variables aléatoires de même espérance finie $m$, et $N$ une variable aléatoire à valeur dans $\mathbb N$ d'espérance finie et indépendante de chacune des variables $(X_k)_{k\geqslant 1}$. Alors la variable aléatoire définie par :
\[
S = \sum_{k=1}^N X_k
\]
est d'espérance finie et $E(S) = E(N)\,m$.
\end{lemme}

\section{Solution du deuxième exercice}

On cherche un polynôme annulateur simple de $H.$

Par un calcul direct, on a pour tout $u\in \mathcal{L}(E),$  $$H^{2}(u)=\frac{1}{4}(u\circ p+p\circ u)+\frac{1}{2}p\circ u \circ p \mbox{ et } 
H^{3}(u)=\frac{1}{8}(u\circ p+p\circ u)+\frac{3}{4}p\circ u \circ p.$$

Ainsi, on obtient : $$2H^{3}-3H^{2}+H=0.$$ 
Et, $H$ annule donc le polynôme $\displaystyle P(X)=2X^{3}-3X^{2}+X=2X(X-\frac{1}{2})(X-1)$ qui est scindé, à racines simples donc $H$ est diagonalisable.