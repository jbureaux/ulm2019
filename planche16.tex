\chapter{Identité sommatoire}

\section{Sujet}

\paragraph{Exercice}
On pose $p_{0}=1$, $p_{1} = 1$ et pour $n\geqslant 2$,
\[
p_{n}=\int_{0}^{x}dx_{1}\int_{0}^{1-x_{1}}dx_{2}\cdots\int_{0}^{1-x_{n-1}}dx_{n}.
\]
Calculer
\[
\sum_{n=0}^\infty p_n \left(\frac\pi6\right)^n.
\]
\paragraph{ Deuxième exercice}
Soit $P =\displaystyle \sum _{k=0}^d a_kX^k \in \mathbb R[X]. $ On note $f_P : \mathbb R[X] \to \mathbb R[X]$ l'application définie par $f_P(Q) = \displaystyle \sum _{k=0}^d a_k Q^{(k)}$ où $Q^{(k)}$ désigne la dérivée $k$-ième de $Q$.Est-ce que $f_P$ est un isomorphisme? Est-ce que $ P\longmapsto f_P$ est un isomorphisme?
\section{Solution de l'exercice}

On définit pour $x\in[0,1],$ $$p_{0}(x)=1;p_{1}(x)=x$$et pour tout $n\geq 2,$ $$p_{n}(x)=\int_{0}^{x}dx_{1}\int_{0}^{1-x_{1}}dx_{2}\ldots\int_{0}^{1-x_{n-1}}dx_{n}.$$
On a par le caractère muet des variables d'intégration, pour tout $x\in[0,1]$ et pour tout $n\geq 0$ : $$p'_{n+1}(x)=p_{n}(1-x).$$

Soit $y$ vérifiant $0<y<1$ (mais on verra à la fin que $\vert y\vert <\frac{\pi}{2}$ convient).
Pour $x\in[0,1],$ considérons $$\phi_{y}(x)=\sum_{n\geq 0}p_{n}(x)y^{n}.$$

En dérivant termes à termes (car la convergence de la série dérivée en $x$ est normale sur $[0,1]$ à $y$ fixé dans le régime prescrit), 
on obtient pour tout $x\in[0,1],$ $$\phi'_{y}(x)=\sum_{n\geq 0}p'_{n}(x)y^{n}=y\phi_{y}(1-x).$$
On a alors pour tout $x\in[0,1],$ $$ \phi''_{y}(x)=-y^{2}\phi_{y}(x).$$

Ainsi, il existe $A,B\in\mathbb{R}$ tels que pour tout $x\in[0,1],$ $$ \phi_{y}(x)=A\cos(xy)+B\sin(xy).$$
Comme $\phi_{y}(0)=1$ et $\phi'_{y}(1)=y,$ on obtient : $A=1$ et $\displaystyle B=\frac{1+\sin(y)}{\cos(y)}.$

La quantité recherchée dans l'énoncé est alors $\displaystyle \phi_{y}(1)=\frac{1+\sin(y)}{\cos(y)}.$
En particulier pour $\displaystyle y=\frac{\pi}{6},$ il vient la quantité voulue $\displaystyle S:=\phi_{\frac{\pi}{6}}(1)=\sqrt{3}.$

\section{ Solution du deuxième exercice}%LOU16
Si $a_0= 0$, alors $f_P(1) =0,\:\:\text{Ker} f_P \neq \{0\},\:\: f_P\:$ n'est pas injectif.

Si $a_0\neq 0.\quad$ Pour $k>d$, posons $a_k =0$ et considérons la suite $(b_k)_{k\in \mathbb N}$ définie par:   $$b_0 = \dfrac 1 {a_0},\quad \forall  k \in \mathbb N^*,\:\: b_k = - \dfrac 1{a_0}\displaystyle \sum _{i=0}^{k-1}a_{k-i}b_{i} $$ 

(de sorte que $\:\:\forall k \in \mathbb N^*,\:\: \displaystyle \sum_{i=0}^k a_{i} b_{k-i} =0$ ) et l'endomorphisme $g_P$ de $\mathbb R [X]$ défini par: $$  \forall Q \in \mathbb R[X], \quad    g_P(Q) = \displaystyle \sum _{k=0}^{+\infty} b_k Q^{(k)},$$
cette somme ne contenant qu'un nombre fini de termes non nuls. Alors:

$\displaystyle \left(f_P \circ g_P \right) (Q) = \sum _{k=0}^{+\infty} a_k \sum _{i=0}^{+\infty} b_i Q^{(k+i)} = \sum _{n=0}^{+\infty} \left( \sum_{k+i=n} a_kb_i \right) Q^{(n)} = Q,\:$ et, de la même façon: $\left( g_P \circ f_P \right) (Q) =Q.$ On a prouvé: $$ \boxed {f_P \:\text{ est un isomorphisme }\: \iff a_0 \neq 0.}$$ 
$ \begin {array} {ccc} \mathbb R[X]& \longrightarrow & \mathcal L(\mathbb R[X]) \\P &\longmapsto & f_P \end{array}$ est un morphisme d'espaces vectoriels, ainsi qu'un morphisme d'anneaux.


Soit $h$ l'élément de $\mathcal L (\mathbb R[X])$ défini par: $\forall Q \in \mathbb R[X],\:\: h(Q) = XQ.$
Ainsi, $\forall P =\displaystyle \sum  _{k=0}^d a_k X^k \in \mathbb R[X],\:\: f_P(1) =a_0\:$ et $ h(1) = X$. Ainsi, 
$P \longmapsto f_P$ n'est pas surjectif et donc:$$\boxed{ P\longmapsto f_P\: \text{n'est pas un isomorphisme de}\: \mathbb R[X]\: \:\text{sur}\:\: \mathcal L(\mathbb R[X]).}$$ 
 
