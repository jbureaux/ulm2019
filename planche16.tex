\chapter{Planche 16}

\section{Sujet}

\paragraph{Exercice}
On pose $p_{0}=1$, $p_{1} = 1$ et pour $n\geqslant 2$,
\[
p_{n}=\int_{0}^{x}dx_{1}\int_{0}^{1-x_{1}}dx_{2}\cdots\int_{0}^{1-x_{n-1}}dx_{n}.
\]
Calculer
\[
\sum_{n=0}^\infty p_n \left(\frac\pi6\right)^n.
\]

\section{Solution de l'exercice}

On définit pour $x\in[0,1],$ $$p_{0}(x)=1;p_{1}(x)=x$$et pour tout $n\geq 2,$ $$p_{n}(x)=\int_{0}^{x}dx_{1}\int_{0}^{1-x_{1}}dx_{2}\ldots\int_{0}^{1-x_{n-1}}dx_{n}.$$
On a par le caractère muet des variables d'intégration, pour tout $x\in[0,1]$ et pour tout $n\geq 0$ : $$p'_{n+1}(x)=p_{n}(1-x).$$

Soit $y$ vérifiant $0<y<1$ (mais on verra à la fin que $\vert y\vert <\frac{\pi}{2}$ convient).
Pour $x\in[0,1],$ considérons $$\phi_{y}(x)=\sum_{n\geq 0}p_{n}(x)y^{n}.$$

En dérivant termes à termes (car la convergence de la série dérivée en $x$ est normale sur $[0,1]$ à $y$ fixé dans le régime prescrit), 
on obtient pour tout $x\in[0,1],$ $$\phi'_{y}(x)=\sum_{n\geq 0}p'_{n}(x)y^{n}=y\phi_{y}(1-x).$$
On a alors pour tout $x\in[0,1],$ $$ \phi''_{y}(x)=-y^{2}\phi_{y}(x).$$

Ainsi, il existe $A,B\in\mathbb{R}$ tels que pour tout $x\in[0,1],$ $$ \phi_{y}(x)=A\cos(xy)+B\sin(xy).$$
Comme $\phi_{y}(0)=1$ et $\phi'_{y}(1)=y,$ on obtient : $A=1$ et $\displaystyle B=\frac{1+\sin(y)}{\cos(y)}.$

La quantité recherchée dans l'énoncé est alors $\displaystyle \phi_{y}(1)=\frac{1+\sin(y)}{\cos(y)}.$
En particulier pour $\displaystyle y=\frac{\pi}{6},$ il vient la quantité voulue $\displaystyle S:=\phi_{\frac{\pi}{6}}(1)=\sqrt{3}.$

\paragraph{Deuxième exercice}
