\chapter{Dérangements et calculs de déterminants}

\section{Sujet}

\paragraph{Exercice}

\section{Solution de l'exercice} %LOU16 : Merci! ^^

\begin{enumerate}
\item On écrit par d\'{e}finition du déterminant :
$$\mbox{det}(A) = \sum_{\sigma\in \mathcal{S}_{n}}\varepsilon(\sigma)\prod_{k=1}^{n}A_{k,\sigma(k)}.$$
Il vient alors par indépendance des entrées de la matrice et par linéarité de l'espérance :
$$\mathbb{E}[\mbox{det}(A)]= p^{n}\sum_{\sigma\in\mathcal{S}_{n}}\varepsilon(\sigma)=\left\{ \begin{array}{l}
p \mbox{ si } n=1\\
0 \mbox{ si } n\geq 2,
\end{array}\right.$$
car pour $n\geq 2,$ il y a autant de permutations paires que d'impaires ($A_{n}$ est un sous groupe d'indice $2$ dans $\mathcal{S}_{n}$).\\

\item La deuxième partie de cet exercice est fondée sur des techniques de dénombrement basées sur des méthodes de séries génératrices. 

Il vient par la formule du produit matriciel puis en prenant l'espérance :
\begin{align*}
\mathbb{E}[\mbox{det}(A^{2})] & = \sum_{\sigma,\tau\in \mathcal{S}_{n}}\varepsilon(\sigma\tau)\mathbb{E}[\prod_{k=1}^{n}A_{k,\sigma(k)}A_{k,\tau_{k}}]\\
& =\sum_{\sigma,\tau\in \mathcal{S}_{n}}\varepsilon(\sigma\tau)\prod_{k=1}^{n}\mathbb{E}[A_{k,\sigma(k)}A_{k,\tau_{k}}]\\
& = n!\sum_{\sigma\in \mathcal{S}_{n}}\varepsilon(\sigma)p^{2n-f(\sigma)}\\
& \mbox{ où } f(\sigma)=\#\{i\in\{1,\ldots,n\}\mbox{ }|\mbox{ } \sigma(i)=i\}\\
&  = n!p^{n}\sum_{k=0}^{n}p^{n-k}\left(f_{k}^{+}-f_{k}^{-}\right)\\
& \mbox{ où } f_{k}^{+}=\#\{\sigma\in \mathcal{A}_{n}\mbox{ }|\mbox{ } f(\sigma)=k\} \mbox{ et } f_{k}^{-}=\#\{\sigma\in \mathcal{S}_{n}\setminus{\mathcal{A}_{n}}\mbox{ }|\mbox{ } f(\sigma)=k\}.
\end{align*}

On remarque ensuite $$f_{k}^{+}=\binom{n}{k}d_{n-k}^{+} \mbox{ et } f_{k}^{-}=\binom{n}{k}d_{n-k}^{-},$$ où $d_{k}^{+}$ (respectivement $d_{k}^{-}$) désigne les dérangements (permutations sans point fixe) pairs (respectivement impairs) de $\mathcal{S}_{k}$ avec les "conventions" : $\displaystyle d_{0}^{+}=1;d_{0}^{-}=0 \mbox{ et } d_{1}^{+}=0;d_{1}^{-}=0.$ 

Ainsi, on peut écrire par symétrie des coefficients binomiaux : $$\mathbb{E}[\mbox{det}(A^{2})] =  n!p^{n}\sum_{k=0}^{n}\binom{n}{k}p^{k}\left(d_{k}^{+}-d_{k}^{-}\right).$$

Or, pour $n\geq 2,$ on a (car il y autant de permutations paires que de permutations impaires) $$\sum_{k=0}^{n}\binom{n}{k}d_{k}^{+}=\frac{n!}{2}=\sum_{k=0}^{n}\binom{n}{k}d_{k}^{-}.$$

On a alors pour $n\geq 2,$ $$\sum_{k=0}^{n}\frac{1}{(n-k)!}\times \frac{d_{k}^{+}-d_{k}^{-}}{k!}=0.$$

En reconnaissant un produit de Cauchy, il vient (en prenant en compte les termes de bord) 
$$\exp(z)\times \sum_{k\geq 0}\frac{d_{k}^{+}-d_{k}^{-}}{k!}z^{k}=1+z$$ et donc par inversion, $$\forall k\in\mathbb{N},\mbox{ } d_{k}^{+}-d_{k}^{-}=(-1)^{k-1}(k-1).$$

On a alors :
\begin{align*}
\mathbb{E}[\mbox{det}(A^{2})] & =  -n!p^{n}\sum_{k=0}^{n}(-1)^{k}\binom{n}{k}p^{k}(k-1)\\
& -n!p^{n}\frac{d}{dz}\left(\sum_{k=0}^{n}\binom{n}{k}p^{k}z^{k-1}\right)_{|z=-1}\\
& = -n!p^{n}\frac{d}{dz}\left(\frac{(1+pz)^{n}}{z}\right)_{|z=-1}\\
& -n!p^{n}\left(-np(1-p)^{n-1}-(1-p)^{n}\right)\\
& =n!p^{n}(1-p)^{n-1}(np+(1-p))\\
& = n!p^{n}q^{n-1}(np+q) \mbox{ où } q=1-p.
\end{align*}

\end{enumerate}

\section{Solution du deuxième exercice}

\begin{itemize}
\item Supposons, par l'absurde, qu'il existe $\varepsilon>0$ et une suite strictement croissante $t$ de points de $\mathbb{R}^{+}$ tendant vers $+\infty$ tels que $\displaystyle \vert y'(t_{n}) \vert \geq \varepsilon.$

Quitte à extraire une sous-suite (et quitte à changer $y'$ en $-y'$), on peut supposer : $\displaystyle y'(t_{n})\geq \varepsilon.$

Mais alors, par uniforme continuité de $y'$ (qui est une somme d'applications uniformément continues : une application limitée (finie) en $+\infty$ est uniformément continue), il existe $\delta>0$ tel que $$\forall a,b\in \mathbb{R}^{+},\mbox{ } \vert a-b\vert \leq \delta \Longrightarrow \vert y'(a)-y'(b)\vert \leq \frac{\varepsilon}{2}.$$

En particulier, par l'inégalité triangulaire renversée, on a pour $n\gg1,$ $$\int_{t_{n}-\delta}^{t_{n}+\delta}y'(t)\geq 2\delta\times \frac{\varepsilon}{2}=\delta\varepsilon.$$

\item Comme $y'$ est $C^{0}$ sur $\mathbb{R}^{+}$ et que $y$ est limitée (finie) en $+\infty,$ on obtient que l'intégrale impropre $\displaystyle \int_{0}^{+\infty}y'(t)dt$ est convergente.

En particulier, on sait que pour tout $n\gg 1,$ il existe $N\gg 1$ tel que pour tout $b\geq a \geq N,$ $$\vert \int_{a}^{b}y'(t)dt\vert \leq \frac{1}{2^{n}}.$$

\item Quitte à extraire de nouveau une sous-suite de $t$, on obtient une sous-suite de $t$ vérifiant :
$$\forall n\geq 1, \int_{t_{\phi(n)}-\delta}^{t_{\phi(n)}+\delta}y'(t)\leq \frac{1}{2^{n}}.$$

Mais alors pour $N\gg 1,$, on obtient la contradiction désirée, à savoir : 
\begin{align*}
N\varepsilon \delta & \leq \sum_{n=1}^{N}\int_{t_{\phi(n)}-\delta}^{t_{\phi(n)}+\delta}y'(t)dt\\
& \leq \sum_{n=1}^{N} \vert \int_{t_{\phi(n)}-\delta}^{t_{\phi(n)}+\delta}y'(t)dt \vert \\
& \leq \sum_{n=1}^{N}\frac{1}{2^{n}}\\
& \leq 1.
\end{align*}

Ainsi, on a montré : $\displaystyle \lim_{t\rightarrow +\infty}y'(t)=0.$

\end{itemize}
