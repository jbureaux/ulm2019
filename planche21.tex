\chapter{Rigidité des queues de distribution}

\section{Sujet}

\paragraph{Exercice}

Soient $X,Y$ deux variables aléatoires à valeurs dans $\mathbb N$, indépendantes et de même loi, telles que
$$
P(X+Y \geqslant x) \underset{x\to\infty}\sim 2\,P(X \geqslant x).
$$
Montrer que
$$
P(X \geqslant x) \underset{x\to\infty}\sim P(X \geqslant x-1).
$$

\section{Solution de l'exercice}

Notons $q(x) = P(X \geqslant x)$ la queue de distribution. Les variables $X,Y$ sont à valeurs positives, donc la probabilité de l'évènement $(X + Y \geqslant x)$ est supérieure ou égale à celle de $(X\geqslant x) \cup (Y \geqslant x)$, qui vaut :
\[
P(X\geqslant x) + P(Y \geqslant x) - P((X\geqslant x) \cap (Y \geqslant x)) = 2q(x) - q(x)^2.
\]
Puisque $P(X+Y \geqslant x) \sim 2q(x)$ et $q(x) \to 0$  lorsque $x \to \infty$, ces deux probabilités sont donc toutes deux équivalentes à $2q(x)$. Autrement dit, la différence des deux :
\[
P((X + Y \geqslant x) \cap (X < x) \cap (Y < x)),
\]
est négligeable par rapport à $q(x)$. Il en va de même \textit{a fortiori} pour
\[
P((X + Y \geqslant x) \cap (X < x) \cap (Y = a)),
\]
quel que soit l'entier naturel $a$. Ceci revient à dire que, lorsque $x \to \infty$,
\[
P((X+Y \geqslant x) \cap (Y = a)) = P((X \geqslant x) \cap (Y = a)) + o(q(x)).
\]
Or $(X+Y \geqslant x) \cap (Y=a) = (X \geqslant x - a) \cap (Y=a)$, donc par indépendance :
\[
q(x-a)\,P(Y = a) = q(x) P(Y = a) + o(q(x)).
\]
S'il existe un entier $a \geqslant 1$ tel que $P(Y = a) \neq 0$, on obtient $q(x - a) \sim q(x)$ et le résultat en découle facilement car
\[
q(x) \leqslant q(x-1) \leqslant q(x-a).
\]
Sinon, les variables $X,Y$ sont presque sûrement égales à $0$ et le résultat est trivial.\\


\paragraph{Remarque : } pour une variante plus quantitative du résultat...\\

%Il suffit de prouver le résultat pour des $x$ entiers car les variables aléatoires en jeu sont à valeurs dans $\mathbb{N}^{*}.$

%On a clairement pur tout $n\gg1,$ $$\mathbb{P}(X\geq n+1)\leq \mathbb{P}(X\geq n).$$

%\begin{enumerate}
    %\item \underline{ On va d'abord supposer que $X$ et $Y$ sont à valeurs dans $\mathbb{N}^{*}.$}


%Soit $\varepsilon>0$ tel que $\varepsilon\ll1.$

%Pour tout $n\gg1,$ on a alors $$\mathbb{P}(X+Y\geq n+1)\leq 2(1+\varepsilon)\mathbb{P}(X\geq n+1) \mbox{ et } \mathbb{P}(X\geq n)\leq \varepsilon.$$ 

%Ensuite, on a en conditionnant pour tout $n\gg1$ :
%\begin{align*}
%\mathbb{P}(X\geq n) & =\mathbb{P}(X\geq n;Y\in\{1,\ldots,n-1\})+\mathbb{P}(X\geq n;Y\geq n)\\
%\mbox{ et } \mathbb{P}(Y\geq n) & =\mathbb{P}(Y\geq n;X\in\{1,\ldots,n-1\})+\mathbb{P}(Y\geq n;X\geq n).
%\end{align*}

%Comme les Va $X$ et $Y$ sont i.i.d, il vient par somme des deux relations précédentes et pour tout $n\gg1$ :
%\begin{align*}
%2\mathbb{P}(X\geq n) & = \mathbb{P}\big([X\geq n;Y\in\{1,\ldots,n-1\}]\cup[Y\geq n;X\in\{1,\ldots,n-1\}] \big)\\
%&+2\mathbb{P}^{2}(X\geq n)\\
%& \leq \mathbb{P}(X+Y\geq n+1)+2\mathbb{P}^{2}(X\geq n)\\
%& \mbox{ car l'union des évènements en jeu est incluse dans } [X+Y\geq n+1]\\ 
%&\leq 2(1+\varepsilon)\mathbb{P}(X\geq n+1) +2\varepsilon \mathbb{P}(X\geq n).
%\end{align*}

%Il vient alors pour tout $n\gg 1,$ $$\mathbb{P}(X\geq n)\leq \frac{1+\varepsilon}{1-\varepsilon}\mathbb{P}(X\geq n+1).$$

%Et ainsi, on a effectivement $\displaystyle \mathbb{P}(X\geq n)\sim_{n\rightarrow +\infty} \mathbb{P}(X\geq n+1).$

%\item \underline{Il reste maintenant à traiter le cas où $X,Y$ sont à valeurs dans $\mathbb{N}.$}

%Il s'agit juste d'observer dans la preuve précédente une "sorte de compensation" (bootstrap).

Soit $\varepsilon>0$ tel que $\varepsilon\ll1$ et vérifiant $\displaystyle P(X=0):=c<1-\varepsilon$ (ce qui est possible si $X$ n'est pas supposée constante, ce dernier cas se traitant immédiatement).

On note alors $\displaystyle l=1-c-\varepsilon>0.$

Pour tout $n\gg1,$ on a alors $$\mathbb{P}(X+Y\geq n+1)\leq 2(1+\varepsilon)\mathbb{P}(X\geq n+1) \mbox{ et } \mathbb{P}(X\geq n)\leq \varepsilon.$$ 

Ensuite, on a en conditionnant pour tout $n\gg1$ :
\begin{align*}
\mathbb{P}(X\geq n) & =\mathbb{P}([X\geq n;Y\in\{1,\ldots,n-1\}]\cup[X\geq n+1;Y=0])\\
& +\mathbb{P}(X=n;Y=0)+\mathbb{P}(X\geq n;Y\geq n)\\
\mbox{ et } \mathbb{P}(Y\geq n) & =\mathbb{P}([Y\geq n;X\in\{1,\ldots,n-1\}]\cup[Y\geq n+1;X=0])\\
& +\mathbb{P}(Y=n;X=0)+\mathbb{P}(Y\geq n;X\geq n).
\end{align*}

Comme les Va $X$ et $Y$ sont i.i.d, il vient par somme des deux relations précédentes et pour tout $n\gg1$ :
\begin{align*}
2\mathbb{P}(X\geq n) & \leq \mathbb{P}(X+Y\geq n+1)+2c\mathbb{P}(X=n)+2\mathbb{P}^{2}(X\geq n)\\
& \mbox{ car l'union des deux évènements incompatibles apparaissant dans chacun }\\
& \mbox{ des deux premiers termes est incluse dans } [X+Y\geq n+1]\\ 
&\leq 2(1+\varepsilon)\mathbb{P}(X\geq n+1) +2\varepsilon \mathbb{P}(X\geq n)+2c\mathbb{P}(X=n).
\end{align*}

Il vient alors pour tout $n\gg 1,$ $$ (1-\varepsilon)\mathbb{P}(X\geq n)\leq (1+\varepsilon)\mathbb{P}(X\geq n+1)+c\mathbb{P}(X=n).$$

Comme $\displaystyle \mathbb{P}(X\geq n)-\mathbb{P}(X\geq n+1)=\mathbb{P}(X=n),$ on obtient ainsi pour tout $n\gg1,$ $$ (1-\varepsilon)\mathbb{P}(X=n)\leq 2\varepsilon\mathbb{P}(X\geq n+1)+c\mathbb{P}(X=n).$$

Il vient alors pour tout $n\gg 1,$ $$ \mathbb{P}(X=n)\leq \frac{2\varepsilon}{l}\mathbb{P}(X\geq n+1).$$

En écrivant $$\mathbb{P}(X\geq n)=\mathbb{P}(X=n)+\mathbb{P}(X\geq n+1)\leq \left(1+\frac{2\varepsilon}{l}\right)\mathbb{P}(X\geq n+1),$$ on obtient effectivement $$\mathbb{P}(X\geq n)\sim \mathbb{P}(X\geq n+1).$$
%\end{enumerate}

\section{Solution du deuxième exercice}

L'exercice fait appel au caractère multiplicatif des fonctions arithmétiques en jeu (cette idée est très utile pour mieux \emph{dévisser} certains problèmes).\\

Notons pour $n\geq 1,$ $\displaystyle g(n)=\sum_{d|n}f(d).$ 

La fonction $g$ est multiplicative i.e. si $a,b\geq 1$ sont tels que $a\wedge b=1$ alors $\displaystyle g(ab)=g(a)g(b)$ (ceci découle du lemme de Gauss).

Soit $p\in \mathbb{P}.$ 

Pour $\alpha\geq 1,$ il suffit alors de calculer $g(p^{\alpha})$ pour déterminer complètement $g$ sur $\mathbb{N}^{*}.$

Or, par définition, on a :
\begin{align*}
g(p^{\alpha}) & =\sum_{k=0}^{\alpha}(-1)^{f(p^{k})}\\
& =\sum_{k=0}^{\alpha}(-1)^{k}\\
& =\left\{ \begin{array}{l}
0 \mbox{ si } \alpha=1 [2]\\
1 \mbox{ si } \alpha=0 [2].
\end{array}\right.
\end{align*}

Ainsi, $g(n)=0$ si l'une des valuations $p$-adiques de $n$ est impair et sinon, $g(n)=1$ (si tous les valuations $p$-adiques de $n$ sont pairs i.e. $n$ est un carré parfait).

