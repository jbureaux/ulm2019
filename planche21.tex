\chapter{Planche 21}

\section{Sujet}

\paragraph{Exercice}

Soient $X,Y$ deux variables aléatoires à valeurs dans $\mathbb N$, indépendantes et de même loi, telles que
$$
P(X+Y \geqslant x) \underset{x\to\infty}\sim 2\,P(X \geqslant x).
$$
Montrer que
$$
P(X \geqslant x) \underset{x\to\infty}\sim P(X \geqslant x-1).
$$

\paragraph{Deuxième exercice}

\section{Solution de l'exercice}

Notons $q(x) = P(X \geqslant x)$ la queue de distribution. Les variables $X,Y$ sont à valeurs positives, donc la probabilité de l'évènement $(X + Y \geqslant x)$ est supérieure ou égale à celle de $(X\geqslant x) \cup (Y \geqslant x)$, qui vaut :
\[
P(X\geqslant x) + P(Y \geqslant x) - P((X\geqslant x) \cap (Y \geqslant x)) = 2q(x) - q(x)^2.
\]
Ces deux probabilités sont donc toutes deux équivalentes à $2q(x)$ lorsque $x \to \infty$ car $q(x) \to 0$. Autrement dit, la probabilité
\[
P((X + Y \geqslant x) \cap (X < x) \cap (Y < x)) 
\]
est négligeable par rapport à $q(x)$ et il en va en particulier de même pour
\[
P((X + Y \geqslant x) \cap (X < x) \cap (Y = a)),
\]
quel que soit l'entier $a$. Ceci revient à dire que
\[
P((X+Y \geqslant x) \cap (Y = a)) = P((X \geqslant x) \cap (Y = a)) + o(q(x)).
\]
Puisque $(X+Y \geqslant x) \cap (Y=a) = (X \geqslant x - a) \cap (Y=a)$, on obtient
\[
q(x-a)\,P(Y = a) = q(x) P(Y = a) + o(q(x)),
\]
par indépendance de $X$ et $Y$. S'il existe un entier $a \geqslant 1$ tel que $P(Y = a) \neq 0$, on obtient $q(x - a) \sim q(x)$ et le résultat en découle facilement car
\[
q(x) \leqslant q(x-1) \leqslant q(x-a).
\]
Sinon, les variables $X,Y$ sont presque sûrement égales à $0$ et le résultat est trivial.