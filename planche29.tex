\chapter{Ping-pong répulsif}

\section{Sujet}

\section{Solution du premier exercice}

%Preuve fausse (à modifier!) On peut prouver que l'ensemble des $u_{0}$ admissible est un ouvert dense de $\mathbb{R}^{n}$ en utilisant le théorème de Baire.

%A cet effet, raisonnons par l'absurde en supposant que pour n'importe quelle condition initiale $u_{0}\in \mathbb{R}^{n},$ on a pour tout $n\in \mathbb{N}^{*}$ qu'il existe une paire d'indices $(i_{n},j_{n})\in \mathbb{N}^{2}$ telle que $i_{n}+j_{n}=n$ et vérifiant que la suite $\displaystyle u_{n}=A^{i_{n}}B^{j_{n}}u_{0}$ est bornée.

%Fixons $N\in \mathbb{N}^{*}$ (les matrices en jeu sont en particulier inversibles). 
%Introduisons alors pour un couple $(i,j)\in \mathbb{N}^{2}$ tel que $i+j\geq 1,$ l'ensemble fermé suivant :   $$C_{i,j}(N)=\left\{u\in \mathbb{R}^{n}\mbox{ }|\mbox{ } \|A^{i}B^{j}u\|\leq N\right\}.$$

%Pour $n\geq 1,$ on sait alors -comme union finie de fermés- que $\displaystyle \bigcup_{i+j=n}C_{i,j}(N)$ est un fermé.
%Comme intersection de fermés, on sait alors que l'ensemble $\displaystyle D(N):=\bigcap_{n\geq 1}\bigcup_{i+j=n}C_{i,j}(N)$ est un fermé.
%Or, par hypothèse, on sait également  $\displaystyle \bigcup_{N\geq 1}D(N)=\mathbb{R}^{n}.$
%Ainsi, par le théorème de Baire, il existe $N_{0}\in\mathbb{N}^{*}$ tel que $\mbox{int}(D(N))\neq \emptyset.$

%On considère alors une boule $B(u_{0},\varepsilon)\subset D(N_{0})$ où $\varepsilon>0$ et $u_{0}\in D(N_{0}).$

%On dispose alors pour tout $n\in \mathbb{N}^{*}$ d'une paire d'indices $(i_{n},j_{n})\in \mathbb{N}^{2}$ telle que $i_{n}+j_{n}=n$ et vérifiant $\displaystyle \|u_{n}=A^{i_{n}}B^{j_{n}}u_{0}\|\leq N_{0}.$

%Une telle boule contient à dilatation et translation près la base canonique de $\mathbb{R}^{n}$ : $\displaystyle (e_{1},\ldots,e_{n})$ vérifiant pour pour tout $k\in \{1,\ldots,n\}$ et pour tout $(i,j)\in \mathbb{N}^{2}$ tel que $i+j\geq 1$ : $$\|A^{i}B^{j}e_{k}\|\leq \frac{2N_{0}}{\varepsilon}.$$

%Il vient alors par l'inégalité d'Hadamard, pour tout $(i,j)\in \mathbb{N}^{2}$ tel que $i+j\geq 1$ :  : $$\mbox{det}(A)^{i}\times \mbox{det}(B)^{j}\leq \left(\frac{2N_{0}}{\varepsilon}\right)^{n}.$$

%Mais alors pour $i+j\gg 1,$ ceci est une contradiction car $\mbox{det}(A),\mbox{det}(B)>1.$



\section{Solution du deuxième exercice}

Par le théorème des valeurs intermédiaires, il existe de proche en proche une suite de points de $[0,1]^{n+1}$ vérifiant : $0=x_{0}<\ldots<x_{n}=1$ telle que pour tout $k\in\{0,\ldots,n\},$ $\displaystyle f(x_{k})=\frac{k}{n}.$

On a d'une part $$\sum_{k=0}^{n-1}\frac{x_{k+1}-x_{k}}{f(x_{k+1})-f(x_{k})}=n\sum_{k=0}^{n-1}\left(x_{k+1}-x_{k}\right)=n(x_{n}-x_{0})=n.$$

D'autre part, par le théorème des accroissements finis, on obtient pour tout $k\in\{0,\ldots,n-1\}$ qu'il existe $y_{k}\in ]x_{k},x_{k+1}[$ tel que $$\frac{f(x_{k+1})-f(x_{k})}{x_{k+1}-x_{k}}=f'(y_{k})\neq 0.$$

Les points $y_{k}$ ainsi générés sont alors deux à deux distincts (car bien séparés).

Il vient alors le résultat désiré, à savoir : $$\sum_{k=0}^{n-1}\frac{1}{f'(y_{k})}=n.$$